\documentclass[book,spanish,a4paper,12pt]{tfg}
\usepackage{graphicx}
\usepackage[utf8
]{inputenc}
\usepackage{enumitem}
\usepackage{float}
\usepackage[spanish]{babel}


% Editar la titulación
\titulacion{Máster en Computación Gráfica, Realidad Virtual y Simulación}

% Editar el título
\title{Herramienta de generación de terreno procedural para Unity}

% Si es una alumna se debe usar
% \authorlabel{Autora}
\authorlabel{Autor}
% Editar el nombre
\author{Rafael Polope Contreras}


% Si hay varios tutores:
% \tutorlabel{Tutores}
% \tutor{Nombre del tutor 1 \\[2mm] Nombre del turor2}
% Si el tutor es masculino:
\tutorlabel{Tutor}
% \tutorlabel{Tutora}
% Editar
\tutor{Álvaro San Juan Cervera}

% Editar: Poner mes y año de la convocatoria de lectura del TFM
\convocatoria{Septiembre 2023}

\begin{document}

% NO QUITAR ESTOS ELEMENTOS
\portada
\cleardoublepage
% \contraportada
% \cleardoublepage
% \declaracion
% \cleardoublepage


% Editar: Resumen en Español (obligatorio)
\begin{resumen}
  Este trabajo presenta una herramienta de generación de terrenos procedurales para videojuegos. Utiliza diferentes tipos de ruido para crear terrenos variados y realistas, aplicando algoritmos de erosión para mejorar la apariencia. Además, asigna colores según la altitud del terreno. La herramienta aprovecha la paralelización con Job System y utiliza Niveles de Detalle (LOD) para optimizar el rendimiento del juego.\\
  \\
  Uno de los objetivos principales del proyecto es crear terrenos continuos, abordando el problema de las inconsistencias entre las diferentes partes del terreno generado. Además, busca superar los desafíos de rendimiento en tiempo real en la generación de terreno.\\
  \\
  \textbf{Palabras Clave:} Generación Procedural, Ruido, Erosión, Colores por Altura, Job System, LOD.
\end{resumen}

\cleardoublepage

\begin{resumen}
  This work presents a procedural terrain generation tool for video games. It uses different types of noise to create diverse and realistic terrains, applying erosion algorithms to enhance their appearance. Additionally, it assigns colors based on the terrain's elevation. The tool leverages parallelization through Job System and utilizes Levels of Detail (LOD) techniques to optimize game performance.\\
  \\
  One of the primary goals of the project is to create seamless terrains, addressing the issue of inconsistencies between different terrain parts generated. Furthermore, it aims to overcome real-time performance challenges in terrain generation.\\
  \\
  \textbf{Keywords:} Procedural Generation, Noise, Erosion, Color by Height, Job System, LOD.
\end{resumen}

\cleardoublepage

% Editar: Agradecimientos (opcional)\documentclass[10pt]{•}
\begin{agradecimientos}
  A Álvaro San Juan, mi tutor en este proyecto, por ofrecerme la oportunidad de desarrollar este proyecto final a pesar de las dificultades presentadas para llevarlo a cabo en último momento.\\
\\
  A todo el equipo de RRHH de U-TAD y de Lãberit, la empresa en la que he estado trabajando durante el curso de este Máster, por haber hecho lo posible para llevar a cabo las prácticas de manera tan entregada y cercana.\\
\\
  A mi familia por soportarme este año especialmente duro, tanto para ellos como para mí, y a mis amigos por tener paciencia y entender el esfuerzo que me ha supuesto la decisión de llevar a cabo este máster y haber estado tan ausente.\\

\end{agradecimientos}
\cleardoublepage

\tableofcontents

\pagestyle{tfg}
\justify

% Las figuras se buscan en el directorio figs

% Cada capítulo está en su propio fichero tex. Ver el directorio tex.

% La bibliografía está dentro del directorio bib

\renewcommand{\thechapter}{I} % Puedes usar "I" en lugar de "0" si prefieres números romanos

% Prefacio
\chapter*{\centering I. GUÍA DE LA MEMORIA}
Este es el prefacio de la memoria. A continuación, se presenta una breve descripción de los capítulos que componen esta memoria:

\begin{description}
  \item[Capítulo 1. INTRODUCCIÓN.] En este capítulo se realiza una introducción a la motivación por la cual se ha llevado a cabo este proyecto, las técnicas más comunes que se utilizan para la generación procedural, el contexto en el que se desarrolla e importancia de este.

  \item[Capítulo 2. PLANTEAMIENTO DEL PROBLEMA.] En este capítulo se indican los obstáculos que plantea la generación procedural, el manejo de la memoria, de la optimización temporal, la continuidad del terreno, así como el realismo de los resultados.
  
  \item[Capítulo 3. OBJETIVOS.] En este capítulo se indican los objetivos que se pretenden alcnazar en el proyecto y cómo se pretende conseguirlos.

  \item[Capítulo 4. ESTADO DEL ARTE.] En este capítulo se analiza el estado del arte de soluciones de generación procedural con objetivos similares a la propuesta de este proyecto tras realizar una revisión bibliográfica.

  \item[Capítulo 5. DESARROLLO DE LA SOLUCIÓN.] En este capítulo se detalla el análisis, diseño e implementación de la todos los componentes, sistemas y algoritmos empleados para la generación de terreno procedimental en Unity.

  \item[Capítulo 6. ANÁLISIS DE RESULTADOS.] En este capítulo se recogen todos los resultados de las pruebas establecidas para la evaluación de los algoritmos que crean los terrenos de la herramienta. Se describe el conjunto de parámetros empleados y las métricas establecidas para la evaluación de la variable de interés y para la validación de los resultados, tanto visuales como de rendimiento. Se analizan los resultados obtenidos y se identifican los factores y limitaciones que han podido influir en ellos.

  \item[Capítulo 7. CONCLUSIONES Y LÍNEAS FUTURAS.] Se recogen las conclusiones extraídas a partir del diseño, desarrollo y experimentación del método propuesto para la elaboración de la herramienta. Se describen futuras líneas de trabajo que complementen el desarrollo alcanzado y se indica la aplicación de este proyecto en el ámbito del desarrollo de contenido multimedia.
\end{description}

\chapter*{\centering II. MEMORIA}

\addcontentsline{toc}{chapter}{I. GUÍA DE LA MEMORIA}
\addcontentsline{toc}{chapter}{II. MEMORIA}

\renewcommand{\thechapter}{\arabic{chapter}}

\chapter{Introducción}
% Contenidos del capítulo.
% Las secciones presentadas son orientativas y no representan
% necesariamente la organización que debe tener este capítulo.

\section{Introducción}

La industria de los videojuegos ha sido testigo de una transformación excepcional en los últimos años. La búsqueda constante de experiencias más inmersivas y visualmente impresionantes ha llevado a los desarrolladores a explorar nuevas formas de crear mundos virtuales. En este contexto, la generación procedural de terrenos se ha convertido en una herramienta esencial para satisfacer las demandas de los jugadores cada vez más exigentes y hambrientos de autenticidad.

Este Trabajo de Fin de Máster (TFM) se adentra en el emocionante campo de la generación procedural de terrenos en Unity, uno de los motores de desarrollo de videojuegos más utilizados tanto por la industria como por desarrolladores independientes. El objetivo principal de este proyecto es diseñar y desarrollar una herramienta avanzada que permita a los creadores de juegos generar terrenos de manera eficiente y convincente.

Los datos disponibles, incluidos los proporcionados por fuentes como Eurostat y otros informes, indican un crecimiento constante en la industria de los videojuegos. A medida que la audiencia se expande, también lo hacen sus expectativas. Los jugadores contemporáneos buscan experiencias de juego que sean únicas y estimulantes. En este contexto, la generación procedural de terrenos se plantea como una solución que puede hacer posible la creación de mundos tanto auténticos como sorprendentes.

La historia de la generación procedural de terrenos comenzó con la implementación de algoritmos basados en ruido, entre los que destaca el renombrado Perlin Noise, desarrollado por Ken Perlin en la década de 1980. A pesar de que estos enfoques ofrecieron resultados impresionantes para su época, a menudo carecían de la autenticidad y el realismo que los jugadores modernos esperan en la actualidad. Con el aumento de la demanda de experiencias de juego más inmersivas y visualmente impactantes, se volvió crucial mejorar las técnicas de generación de terrenos.

Un avance destacado en este ámbito fue la introducción del Simplex Noise, una mejora significativa respecto al Perlin Noise, que mantuvo la capacidad de generar terrenos naturales, pero con un rendimiento más eficiente. No obstante, lo que realmente marcó un hito en la generación procedural de terrenos fue la incorporación de algoritmos de erosión. Estos algoritmos simulan procesos geológicos y climáticos, lo que resulta en terrenos con características naturales más convincentes, como montañas, cañones y ríos. Esta adición permitió crear mundos virtuales más realistas y creíbles, lo que contribuyó a elevar la calidad general de los juegos.

La creciente popularidad de los mundos abiertos en los videojuegos presentó un desafío significativo: generar terrenos continuos y sin interrupciones notables cuando los jugadores exploran los límites del mundo virtual. A medida que los algoritmos mejoraron, los juegos pudieron ofrecer experiencias de juego más fluidas y expansivas, lo que aumentó la inmersión del jugador y su sensación de exploración sin restricciones.

La influencia de la generación procedural de terrenos no se limita únicamente a la creación de paisajes virtuales. Ha dejado una huella indeleble en otros aspectos de la generación procedural en la industria de los videojuegos. Esto incluye la generación de ciudades completas en juegos de mundo abierto, la creación de misiones y contenido diverso que aumenta la rejugabilidad, así como la generación de personajes y criaturas que dan vida a los mundos virtuales. Además, ha influido en la generación de texturas y elementos artísticos, permitiendo un ahorro significativo de tiempo y recursos en el desarrollo de juegos y animaciones. La música y los efectos de sonido también se benefician de la generación procedural, adaptando la banda sonora y la atmósfera del juego en tiempo real para acompañar la acción.

En la última década, hemos sido testigos de un aumento significativo en la adopción de la generación procedural de terrenos en la industria de los videojuegos. Esto marca un cambio notable en la forma en que los desarrolladores abordan la creación de entornos virtuales. Esta adopción creciente se ha visto impulsada por varios factores clave, como la capacidad de generar mundos virtualmente infinitos sin incurrir en costos prohibitivos de almacenamiento. Además, proporciona una experiencia de juego única en cada partida, lo que aumenta la rejugabilidad y la longevidad de un título. Esta estrategia también es esencial en la creación de mundos abiertos sin pantallas de carga notables entre escenarios o biomas, lo que mejora la inmersión del jugador.

La evolución de la generación procedural de terrenos ha estado intrínsecamente ligada a la constante innovación en algoritmos y técnicas. En los últimos años, se han desarrollado y refinado una variedad de enfoques que van desde la generación basada en ruido, como el Perlin Noise y el Simplex Noise, hasta técnicas más avanzadas que incorporan algoritmos de erosión y simulaciones físicas para lograr terrenos aún más realistas.

Uno de los desafíos cruciales en la generación procedural de terrenos, especialmente en entornos de mundo abierto, es la gestión eficiente de los "chunks" de terreno, pequeñas porciones de terreno que se generan y almacenan en memoria para luego ser cargadas y descargadas dinámicamente mientras el jugador se mueve a través del mundo virtual. Este proceso es esencial para mantener un rendimiento óptimo y evitar la carga excesiva de recursos en la memoria del sistema. Además, es fundamental asegurarse de que la transición entre diferentes chunks sea fluida y sin discontinuidades notorias, lo que garantiza una experiencia de juego inmersiva. En el contexto de la generación procedural de terrenos, se ha consolidado una técnica comúnmente empleada denominada "sistema de streaming" de terrenos. Este enfoque se encarga de gestionar la carga y descarga de fragmentos o "chunks" de terreno de manera dinámica en función de diversos factores, como la proximidad del jugador, la dirección de su movimiento y su campo de visión. Uno de los principales objetivos de este sistema es garantizar una transición fluida entre chunks adyacentes, evitando discontinuidades notorias que puedan afectar negativamente a la cohesión y la inmersión en el mundo virtual.

Además, se emplean algoritmos de detección de colisiones y de visibilidad para determinar qué chunks deben ser cargados y renderizados según la posición actual de la cámara del jugador. Estos algoritmos son cruciales para optimizar el rendimiento del juego, ya que permiten reducir la carga en la unidad central de procesamiento (CPU) y la unidad de procesamiento gráfico (GPU). En consecuencia, solo se procesan y muestran los chunks que son relevantes y visibles desde la perspectiva del jugador en ese momento particular. Esta estrategia de optimización contribuye significativamente a lograr una experiencia de juego fluida y eficiente en términos de recursos.

En lo que respecta al movimiento de la cámara del jugador y su influencia en la generación del terreno, se implementan técnicas de "nivel de detalle" (LOD, por sus siglas en inglés) dinámico. Esta técnica se encarga de ajustar la resolución y el nivel de detalle del terreno en tiempo real en función de la distancia entre la cámara y el terreno circundante. Cuando la cámara se acerca a un chunk de terreno en particular, se aumenta el nivel de detalle, lo que implica mostrar texturas de mayor calidad y una geometría más detallada. Por otro lado, cuando la cámara se aleja, se reduce el nivel de detalle para conservar recursos de hardware y mantener un rendimiento óptimo.

En este contexto, el uso del Job System de Unity ha ganado relevancia. Permite la optimización de procesos intensivos en CPU, como la modificación de mallas de terreno, lo que resulta en una mejora significativa del rendimiento en juegos que implementan generación procedural de terrenos.

\section{Motivación}

El desarrollo de la herramienta de generación procedural de terrenos en Unity se enmarca en un contexto dinámico y desafiante, impulsado por diversas motivaciones que abordan necesidades y aspiraciones cruciales.

La industria de los videojuegos se encuentra en constante evolución, y la búsqueda incesante de experiencias más inmersivas y visualmente impactantes impulsa a los desarrolladores a explorar nuevas formas de crear mundos virtuales. La generación procedural de terrenos se erige como una respuesta a esta demanda creciente. Esta técnica no solo permite satisfacer las expectativas de los jugadores modernos en términos de autenticidad y sorpresa, sino que también agiliza el proceso de desarrollo de videojuegos al proporcionar herramientas eficientes para crear entornos expansivos y convincentes.

La gestión eficiente de chunks de terreno, la optimización del "Job System" de Unity y la elección de algoritmos adecuados son áreas que requieren innovación y soluciones creativas. Esta necesidad de abordar y superar desafíos técnicos en el campo de la generación procedural de terrenos motiva el desarrollo de esta herramienta. La oportunidad de enfrentar estos desafíos y crear una herramienta que marque la diferencia en la industria es una fuerza impulsora clave.

Este proyecto se enmarca en el contexto de una industria de videojuegos en constante cambio y una demanda creciente de experiencias inmersivas. La búsqueda de soluciones técnicas innovadoras y la aspiración de contribuir al crecimiento de la comunidad de desarrolladores son las motivaciones que guían este esfuerzo. La generación procedural de terrenos en Unity tiene el potencial de transformar la forma en que se crean mundos virtuales, y este proyecto se esfuerza por lograr precisamente eso.
\newpage


\chapter{Planteamiento del problema}
\section{Problemática en la Generación Procedural de Terreno}

En este trabajo se pretende mejorar la eficiencia de los videojuegos y la experiencia de los jugadores a través de soluciones destinadas a solucionar problemas relacionados con la generación procedural de terreno en Unity.

De este modo, la gestión de chunks de terreno en el apartado de memoria es fundamental para mejorar la eficiencia, consiguiendo una transición suave según el movimiento de la cámara del jugador y su respectiva proximidad con respecto al entorno. Para conseguir una transición visual adecuada se ha realizado un sistema de "streaming" de terrenos, el cuál carga y descarga los chunks de una manera dinámica mediante técnicas de interpolación y suavizado en las fronteras de los chunks.

Otro objetivo pasa por integrar de la mejor manera el "Job System" de Unity. Aunque no resulta sencillo compaginarlo con otros sistemas del juego y surge la problemática de gestionar distintos trabajos en paralelo, este sistema resulta clave a la hora de mejorar el rendimiento general.

Por otro lado, si bien es cierto que la generación de terreno puede llevarse a cabo correctamente, en ningún momento debe descuidarse el uso de hardware y el tiempo de procesamiento. Para minimizar ambas variables es necesario seleccionar los algoritmos adecuados y optimizarlos correctamente.

Otro desafío pasa por la formación de desarrolladores en el ámbito de la generación de terreno. Para facilitar un aprendizaje adecuado resulta óptimo suministrar herramientas de aprendizaje, que ayuden a los programadores a implementar en sus proyectos técnicas avanzadas.

Aplicar los objetivos mencionados resulta esencial en la mayoría de proyectos, sobre todo en aquellos que utilicen muchos recursos. De este modo, la mejora fundamental pasa por la eficiencia de los algoritmos en función de memoria y rendimiento, la correcta integración de efectos en las escenas y su facilidad y versatilidad de uso.

% \newpage

\chapter{Objetivos}
% Contenidos del capítulo.
% Las secciones presentadas son orientativas y no representan
% necesariamente la organización que debe tener este capítulo.


\section{Objetivos Generales y Específicos}

Este proyecto de generación procedural de terrenos en Unity se enfrenta al desafío de diseñar y desarrollar una herramienta innovadora que aborde las necesidades de la industria de los videojuegos. Los objetivos generales y específicos se centran en la creación de una solución tecnológica efectiva y en la evaluación de su impacto en el proceso de desarrollo de videojuegos. Aquí se desarrollan los objetivos que se deben lograr para llevar a cabo la herramienta conforme se desea:

\begin{enumerate}
    \item \textbf{Diseñar y Desarrollar de la Herramienta:} El objetivo fundamental es crear una solución tecnológica efectiva que permita a los desarrolladores de videojuegos generar terrenos de manera eficiente y convincente. Esta herramienta debe ser capaz de abordar los desafíos técnicos de la generación procedural de terrenos y proporcionar una interfaz intuitiva para los usuarios. 

    \item \textbf{Analizar el Impacto y la Eficiencia de la Herramienta:} Más allá de su desarrollo, es esencial evaluar el aporte real de esta herramienta en el proceso de creación de videojuegos. Esto implica medir la capacidad de la herramienta para optimizar la generación de terrenos y su influencia en la calidad de los juegos resultantes.

    \begin{enumerate}[label=\Alph*)]
        \item \textbf{Investigación y Selección de Algoritmos y Técnicas:} Se realizará una investigación exhaustiva para analizar y seleccionar los algoritmos y técnicas más adecuados para la generación procedural de terrenos en Unity.

        \item \textbf{Desarrollo de la Interfaz y Experiencia de Usuario:} El diseño de la interfaz de usuario y la experiencia de usuario son elementos cruciales para la herramienta. Se aplicará una estrategia de diseño centrada en el usuario.

        \item \textbf{Optimización del Rendimiento:} La generación procedural de terrenos puede ser intensiva en términos de rendimiento. Se establecerán estrategias de optimización para garantizar que la herramienta funcione de manera eficiente en diferentes entornos de desarrollo.

        \item \textbf{Evaluación de la Herramienta en un Entorno Real:} Se llevará a cabo un piloto experimental para evaluar la eficacia y eficiencia de la herramienta en un contexto de desarrollo de videojuegos.
    \end{enumerate}
\end{enumerate}

\newpage



\chapter{Estado del arte}
\section{Introducción}

La generación automática de paisajes en el mundo de los videojuegos ha experimentado una evolución constante debido a la creciente demanda de experiencias de juego más envolventes y visuales impactantes. En este documento, exploramos esta técnica en detalle, con un enfoque particular en su implementación en el motor de desarrollo de videojuegos Unity.

La generación automática de paisajes se ha convertido en una herramienta esencial para los desarrolladores de videojuegos, ya que les permite diseñar mundos virtuales expansivos y auténticos que satisfacen las expectativas cada vez más altas de los jugadores. Este proyecto se centra en las técnicas, algoritmos y herramientas que han impulsado esta disciplina en la última década. A continuación, desglosaremos los puntos clave de este documento:

\subsection{Contexto Histórico}

Para comprender la importancia de la generación automática de paisajes, es fundamental contextualizar su evolución a lo largo del tiempo. Comenzaremos con una breve mirada a su historia, examinando cómo ha avanzado desde sus modestos inicios hasta convertirse en un elemento esencial en la creación de mundos virtuales de alta calidad.

\subsection{Técnicas y Algoritmos}

Un aspecto fundamental en la generación de paisajes automáticos son las técnicas y algoritmos que respaldan su funcionamiento. Exploraremos detalladamente algunos de los enfoques más influyentes y ampliamente utilizados, desde los algoritmos basados en ruido, como el Perlin Noise, hasta técnicas más avanzadas que incorporan simulaciones físicas y procesos geológicos para lograr paisajes extremadamente realistas.

\subsection{Herramientas y Recursos en Unity}

Unity, uno de los motores de desarrollo de videojuegos más populares, proporciona a los desarrolladores herramientas y recursos para la generación procedural de terreno. En esta revisión, examinaremos cómo Unity simplifica la creación de paisajes de manera eficiente y convincente, destacando algunos de los complementos y extensiones más notables que facilitan aún más este proceso.

\subsection{Antecedentes en Unity}

En esta sección, proporcionaremos una breve descripción de los antecedentes relacionados con la generación automática de paisajes en Unity. Exploraremos en detalle las herramientas y recursos disponibles en Unity para la generación de paisajes automáticos, analizando diversas soluciones, como complementos, activos y técnicas específicas que simplifican la creación y manipulación de paisajes dentro de la plataforma Unity.

\subsection{Desafíos y Futuras Tendencias}

Dado que la generación automática de paisajes se ha vuelto más común, también ha enfrentado desafíos significativos. Exploraremos las dificultades más comunes, como la gestión de chunks de terreno en la memoria. Además, consideraremos las futuras tendencias y avances que podrían dar forma al desarrollo de esta rama en los próximos años.

\section{Definición del Tema}

La generación automática de paisajes es una disciplina informática que ha desempeñado un papel fundamental en la creación de mundos virtuales, especialmente en la industria de los videojuegos. En esencia, se refiere a la creación automática y algorítmica de entornos de paisaje en mundos virtuales.

La generación automática de paisajes se basa en algoritmos matemáticos y computacionales y estos algoritmos se basan en principios como el ruido, la interpolación y la simulación de procesos naturales para crear paisajes realistas y convincentes.

La generación de paisajes se lleva a cabo mediante la manipulación de datos y la aplicación de fórmulas matemáticas para determinar las alturas y texturas de cada punto del paisaje, 

La generación automática de paisajes desempeña un papel crucial en la industria de los videojuegos, donde se utiliza para crear mundos expansivos y auténticos. Los juegos de mundo abierto utilizan usualmente esta técnica, ya que les permite ofrecer entornos de gran tamaño sin pantallas de carga notables, mejorando la inmersión del jugador, pero su influencia se extiende más allá de los videojuegos. Se utiliza en aplicaciones de simulación y de  visualización arquitectónica, donde la creación de entornos realistas es esencial.

A medida que la tecnología avanza, los enfoques modernos de generación automática de paisajes han incorporado algoritmos más avanzados, como los que simulan procesos geológicos y climáticos. Además, se están explorando técnicas que permiten una mayor interacción y personalización de los paisajes por parte de los jugadores.

\section{Historia y Evolución de la Generación Procedural de Terrenos}

La generación procedural de terrenos es una técnica importante en la industria de los videojuegos y otros campos. Su principal ventaja radica en la capacidad de crear mundos y contenidos de manera dinámica, sin la necesidad de almacenar grandes cantidades de datos en el disco duro. Con el tiempo, esta técnica ha evolucionado significativamente, impulsada por avances en el hardware y la creciente demanda de mundos más expansivos y realistas  \cite{HistoryofComputerAnimation}.

\subsection{Década de los 60}

La generación procedural de terrenos en gráficos por computadora comenzó a surgir en la década de 1960. En aquel entonces, los gráficos por computadora se utilizaban principalmente con fines científicos, de ingeniería e investigación, aunque también se iniciaron experimentos artísticos \cite{ComputerGraphicsLearningMaterials}. Las técnicas empleadas para la generación procedural de terrenos se agrupan en la categoría más amplia de generación procedural, que implica generar objetos o valores específicos a través de algoritmos. Un ejemplo temprano de generación procedural en gráficos por computadora fue la transformación de declaraciones matemáticas en vectores de herramientas de máquinas en 3D generados por computadora, desarrollado por Douglas T. Ross en 1959 \cite{ComputerGraphics}.

Sin embargo, no fue hasta mediados de la década de 1960 que comenzaron a aparecer experimentos artísticos, especialmente de la mano del Dr. Thomas Calvert. En cuanto a la generación específica de terrenos, un artículo de G.S. Miller titulado "La definición y representación de mapas de terreno" se presentó en la 13ª Conferencia Anual sobre Gráficos por Computadora y Técnicas Interactivas en 1986. Este artículo discutió el uso de algoritmos fractales para la generación y representación de terrenos, así como la generación de terrenos procedurales en tiempo real, que también se discutió en un artículo de 2004 de J. Olsen titulado "Generación de Terrenos Procedurales en Tiempo Real" \cite{SurveyProceduralWorlds}.

\subsection{Década de los 80}

En la década de 1980, la generación procedural de terrenos en gráficos por computadora continuó evolucionando, impulsada por avances tecnológicos y la creciente popularidad de los videojuegos. Aunque la información específica sobre esta década es limitada en los resultados de búsqueda, se pueden inferir algunos desarrollos basados en el progreso general de la generación procedural durante este período.

Uno de los primeros ejemplos de generación procedural en videojuegos se puede rastrear hasta el género de juegos de rol de mesa (RPG). El sistema de juego de mesa líder en ese momento, Advanced Dungeons \& Dragons, proporcionaba formas para que el "maestro de mazmorras" generara mazmorras y terrenos utilizando tiradas de dados aleatorias y tablas procedimentales de ramificación complejas. Este concepto luego se adaptó a los juegos de computadora, con Strategic Simulations lanzando el Dungeon Master's Assistant, un programa que generaba mazmorras basadas en las tablas publicadas \cite{SurveyProceduralWorlds}. En el contexto de gráficos por computadora y videojuegos, la generación procedural de terrenos se convirtió en una herramienta valiosa para crear paisajes realistas y diversos, especialmente en juegos de mundo abierto que requerían entornos vastos y detallados. El uso de algoritmos para generar terrenos permitió a los desarrolladores reducir la cantidad de trabajo manual y crear paisajes que parecían infinitos en tamaño \cite{ProceduralWorlds}. Uno de los primeros métodos para la generación procedural de terrenos fue el algoritmo de diamante-cuadrado, una técnica de modelado fractal simple. Este algoritmo permitía la generación de modelos de terreno altamente detallados al subdividir iterativamente un cuadrado y ajustar los valores de altura en cada paso \cite{ComputerGraphics}.

\subsection{Década de los 2000}

En la década de 2000, la generación procedural de terrenos en gráficos por computadora continuó avanzando, impulsada por la creciente popularidad de los videojuegos y la necesidad de entornos más realistas y diversos. El uso de la generación procedural en videojuegos se volvió más común durante esta década, con muchos juegos generando aspectos del entorno o personajes no jugadores de manera procedural durante el proceso de desarrollo para ahorrar tiempo en la creación de assets \cite{TexturingModelingProcAproach}. Uno de los avances destacados en la generación procedural de terrenos durante esta década fue el desarrollo de nuevos algoritmos y técnicas. Por ejemplo, se presentó un nuevo método para la generación procedural de terrenos en un artículo de 2015 escrito por Christian Schulte titulado "A Graph-Based Approach to Procedural Terrain". Este artículo describe un proceso de tres pasos para generar terrenos utilizando un enfoque basado en grafos. Otro desarrollo notable durante esta década fue el uso de la generación procedural como una mecánica de juego, como la creación de nuevos entornos para que los jugadores los exploraran. Por ejemplo, los niveles en el juego Spelunky se generan de manera procedural al reorganizar mosaicos prefabricados de geometría en un nivel con una entrada, una salida, un camino resoluble entre los dos y obstáculos en ese camino \cite{ProceduralTerrainAproach}.

En resumen, la década de 2000 vio un progreso continuo en la generación procedural de terrenos en gráficos por computadora, con avances en algoritmos y técnicas, así como un aumento en el uso de la generación procedural como una mecánica de juego.

\subsection{Década de los 2010}

En la década de 2010, la generación procedural de terrenos en gráficos por computadora continuó avanzando, con el desarrollo de nuevos algoritmos y técnicas para la generación de terrenos. El uso de la generación procedural en videojuegos también siguió creciendo, con muchos juegos utilizando la generación procedural para crear entornos vastos y diversos.

Un ejemplo de un nuevo algoritmo para la generación procedural de terrenos en la década de 2010 es el algoritmo adaptativo y de generación procedural de terrenos con modelos de difusión y ruido de Perlin propuesto en un artículo de 2021 por Zhang et al. Este algoritmo utiliza modelos generativos basados en difusión para crear terrenos con múltiples niveles de detalle, lo que permite una generación más eficiente de entornos a gran escala \cite{AdaptiveandMulti-resolutionProcedurtalInfiniteTerrain}. Otro ejemplo de generación procedural de terrenos en la década de 2010 es la generación procedural de carreteras, propuesta en un artículo de 2010 por Galin. Este método utiliza un algoritmo de ruta más corta anisotrópica ponderada para generar carreteras automáticamente, lo que permite la creación más eficiente de redes de carreteras en entornos a gran escala \cite{ProceduralGenerationRoads}.

En cuanto a los gráficos, se encontró una publicación de blog sobre la generación procedural de terrenos utilizando Unity. El artículo describe un proyecto que utiliza generación de malla y ruido, así como simulación de erosión hidráulica, para crear terrenos realistas en tiempo real \cite{ProceduralTerrainGeneration}.

En definitiva, se podría decir que la década de 2010 vio un progreso continuo en la generación procedural de terrenos en gráficos por computadora, con el desarrollo de nuevos algoritmos y técnicas, y el aumento en el uso de la generación procedural en videojuegos para crear entornos vastos y diversos.

\subsection{El Futuro de la Generación Procedural} 

El futuro de la generación procedural de terrenos en gráficos por computadora probablemente estará moldeado por avances en redes neuronales, un mayor uso de la generación de activos procedural y la integración de datos del mundo real. Las redes neuronales tienen el potencial de mejorar el realismo y la complejidad de los terrenos generados de manera procedural. Al utilizar técnicas de transferencia de estilo, los desarrolladores pueden crear formas generales y permitir que la red neuronal agregue detalles que parecen realistas \cite{StyleTransfer}. La generación de activos procedural, que utiliza algoritmos para crear automáticamente activos como modelos 3D y texturas, también puede desempeñar un papel significativo en el futuro de la generación de terrenos \cite{RealWorldDataTerrain}. Al combinar diversas técnicas de generación procedural sintética con modelos digitales de elevación (DEM) y datos del mundo real, los desarrolladores pueden crear paisajes multibioma con mayor precisión y atractivo visual. La integración de datos del mundo real, como imágenes de satélite y mapas topográficos, puede mejorar aún más el realismo y la precisión de los terrenos generados de manera procedural. Al combinar estas fuentes de datos con técnicas de generación procedural, los desarrolladores pueden crear entornos más inmersivos y detallados \cite{ProceduralTerrainGenerator}.

En cuanto a hardware y rendimiento, ya se ha explorado el uso de GPU para generar terrenos procedurales complejos a velocidades de fotogramas interactivas \cite{proceduralTerrainGpu}. A medida que la tecnología continúa avanzando, podemos esperar más optimizaciones y mejoras en la eficiencia de los algoritmos de generación procedural de terrenos, lo que permitirá entornos aún más detallados y realistas.

El futuro de la generación procedural de terrenos en gráficos por computadora probablemente estará caracterizado por la combinación de redes neuronales, generación de activos procedural, integración de datos del mundo real y avances en hardware y rendimiento. Estos desarrollos permitirán a los desarrolladores crear entornos más inmersivos y diversos para videojuegos, simulaciones y otras aplicaciones.


\section{Técnicas y Algoritmos}

Como ejemplo de técnica tenemos el ruido de Perlin, el ruido simplex o por ejemplo el ruido de Voronoi. Todos ellos se asocian normalmente a la generación procedural de terrenos y todas estas técnicas se centran en generar números aleatorios dentro de un rango que comúnmente es [-1,1] y con estos números que se generan se utilizan después para crear normales usadas para representar en un mapa y de ese modo generar el terreno. Combinando diferentes funciones, podremos crear paisajes más complejos y evitar también ver los posibles patrones que puedan generar.

Las funciones de ruido, como el ruido de Perlin, el ruido Simplex o el ruido Voronoi, se utilizan comúnmente en la generación procedural de terrenos. Estas funciones generan valores aleatorios en el rango [-1,1] que se pueden usar para crear mapas de alturas normalizando los valores, que luego se utilizan para generar el terreno. Al combinar diferentes funciones de ruido, los desarrolladores pueden crear paisajes más complejos y diversos.

Otro tipo de algoritmos son los de tipo fractal como por ejemplo el algoritmo diamante-cuadrado. Esta tipología utiliza divisiones recursivas para generar terrenos agregando en cada una de estas subdivisiones más nivel de detalle.

Tenemos también enfoques basados en grafos como el que presenta Christian Schulte \cite{ProceduralTerrainGeneration} en 2015, donde utiliza grafos para representar y manipular el terreno. Con este enfoque logró agregar características más complejas como ríos o carreteras.

Existen otras técnicas para la generación de terrenos procedural que incluye simulación de elementos atmosféricos como la erosión causada por el agua o el viento. Para ello utiliza datos del mundo real como imágenes de satélite o mapas topográficos, creando terrenos más precisos y realistas.

\subsection{Generación Procedural de Terrenos basada en Funciones de Ruido}

En el mundo de los videojuegos la generación procedural usando funciones de ruido es una técnica bastante habitual. Como hemos visto en el punto anterior, mediante el uso de algoritmos como Perlin o Simplex, generamos valores aleatorios que utilizaremos en estos mapas como valor de altura y de ese modo tener el terreno.

Combinando distintas funciones de ruido podemos crear mapas de modo relativamente sencillo y con poco trabajo manual, lo cual es una ventaja. Combinado estas funciones los programadores crean las colinas, montañas, etc…

Algunas de las funciones de ruido más comúnmente utilizadas para la generación procedural de terrenos incluyen:
\begin{itemize}
    \item \textbf{Ruido de Perlin:} El ruido de Perlin es un tipo de ruido de gradiente desarrollado por Ken Perlin en la década de 1980. Es un ruido suave y continuo que permite transiciones graduales entre diferentes valores \cite{perlinnoise}.
    
    \item \textbf{Ruido Simplex:} El ruido Simplex es un tipo de ruido de gradiente desarrollado por Ken Perlin en 2001. Es similar al ruido de Perlin pero es más rápido y tiene una mejor calidad visual, dando lugar a terrenos con caracterísitcas realistas. Es comúnmente utilizado.\cite{simplexnoise}.
    
    \item \textbf{Ruido Voronoi:} El ruido Voronoi es un tipo de ruido celular que se genera dividiendo el espacio en celdas basadas en la distancia a un conjunto de puntos de referencia. A menudo se utiliza para generar terrenos con características más geométricas debido a sus aspecto regular. \cite{voronoinoise}.
\end{itemize}

\subsection{Algoritmos basados en fractales para la generación de terrenos procedurales}

La generación procedural de terrenos a menudo se basa en algoritmos fractales que permiten crear paisajes realistas. Algunos de los algoritmos fractales más populares para la generación procedural de terrenos incluyen:

\begin{itemize}
    \item \textbf{Algoritmo Diamante-Cuadrado}: El algoritmo Diamante-Cuadrado es un método simple y eficiente para generar terreno fractal. Comienza con una cuadrícula cuadrada y, en cada iteración, divide cada cuadrado en cuatro cuadrados más pequeños. La altura de los nuevos vértices se ajusta en función del promedio de los vértices originales. Este proceso se repite varias veces, lo que da como resultado un terreno con una apariencia fractal \cite{DiamanteCuadrado}.
    
    \item \textbf{Algoritmo de Desplazamiento del Punto Medio}: El algoritmo de Desplazamiento del Punto Medio es otro método simple para generar terreno fractal. Comienza con un segmento de línea y, en cada iteración, divide el segmento en dos partes. La altura de los nuevos vértices se ajusta mediante un valor de desplazamiento aleatorio. Este proceso se repite, creando un terreno con características fractales \cite{DesplazamientoPuntoMedio}.
    
    \item \textbf{Movimiento Browniano Fractal (FBM)}: El Movimiento Browniano Fractal (FBM) es una técnica que combina múltiples capas de funciones de ruido para crear terrenos más complejos y detallados. A menudo se utiliza en conjunto con otros algoritmos fractales, como el Diamante-Cuadrado o el Desplazamiento del Punto Medio, para generar terrenos realistas y visualmente atrassets \cite{TFractionalBrownianMotion}.
    
    \item \textbf{Técnicas multifractales}: Las técnicas multifractales utilizan diferentes dimensiones fractales para diferentes escalas, lo que permite una representación más precisa del comportamiento del espectro de frecuencias de los paisajes reales. Estas técnicas se pueden utilizar para generar terreno con una apariencia más realista y diversa \cite{calvet2008multifractal}.
    
    \item \textbf{Enfoques híbridos}: Además de los algoritmos fractales puros, existen enfoques híbridos que combinan técnicas fractales con otros métodos, como simulaciones de erosión o modelos geológicos. Estos enfoques pueden utilizarse para generar terrenos más realistas y visualmente atrassets.
\end{itemize}

Los desarrolladores pueden emplear algortimos basados en fractales junto con otras técnicas para generar terrenos con caracterísitcas únicas, permitiendo así ampliar el abanico de posibildades para los terrenos generables.

\subsection{Algoritmos de Simulación Física en la Generación Procedural de Terrenos}

Los algoritmos de simulación física pueden utilizarse en la generación procedural de terrenos para crear terrenos más realistas y de aspecto natural. Estos algoritmos simulan los efectos de procesos físicos como la erosión, la deposición y la meteorización en el terreno, lo que resulta en un terreno que parece haber sido moldeado por las fuerzas naturales, como el viento, la lluvia o el calor. Algunos de los algoritmos de simulación física más comúnmente utilizados en la generación procedural de terrenos incluyen:

\begin{itemize}
    \item \textbf{Algoritmos basados en hidrología}: Los algoritmos basados en hidrología simulan el flujo del agua sobre el terreno, teniendo en cuenta factores como la pendiente, la lluvia y la evaporación. Estos algoritmos pueden utilizarse para crear terrenos con redes de ríos realistas, lagos y otras características de agua \cite{AlgoritmosHidrologia} \cite{HidrologiaDocumento}.
    
    \item \textbf{Algoritmos de simulación de erosión}: Los algoritmos de simulación de erosión simulan los efectos de la erosión causada por el agua y el viento en el terreno, lo que da como resultado terrenos con características realistas como valles, crestas y cañones. Estos algoritmos pueden utilizarse en conjunto con otras técnicas de generación procedural, como funciones de ruido o fractales, para crear terrenos realistas y de aspecto natural \cite{AlgoritmosErosion} \cite{ErosionReddit}.
    
    \item \textbf{Algoritmos de modelado geológico}: Los algoritmos de modelado geológico simulan los procesos geológicos que dan forma al terreno, como la actividad tectónica y las erupciones volcánicas. Estos algoritmos pueden utilizarse para crear terrenos con características geológicas realistas, como montañas, volcanes y líneas de falla. aunque sulen dar resultados realistas a menuido se deprecan por el coste computacional que suponen \cite{GeologiaDocumento}.
\end{itemize}

Al simular los efectos de procesos físicos en el terreno, los desarrolladores pueden crear terrenos que parecen haber sido moldeados por fuerzas naturales, lo que resulta en un entorno más inmersivo y creíble. 

Además de los ya mencionados, se siguen desarrollando nuevas téncicas y nuevos algoritmos de simulación física para la generación procedural de terrenos se centran en mejorar la realismo, eficiencia y flexibilidad en la creación de terrenos. Algunos de los avances en este campo incluyen:

\begin{itemize}
    \item \textbf{Simulación Física en GPU}: Utilizando la potencia de procesamiento paralelo de las GPUs modernas, los investigadores han desarrollado técnicas para generar terrenos procedurales complejos en tiempo real. Estas técnicas aprovechan las capacidades de las GPUs, como el geometry shader, stream output y el renderizado a texturas 3D, para generar rápidamente grandes bloques de terreno detallado\cite{SimulacionFisicaGPU}.
    
    \item \textbf{Algoritmos Inspirados en la Hidrología}: Basándose en el concepto de generación de terrenos basada en hidrología, los algoritmos más recientes incorporan modelos de flujo de agua y erosión más realistas. Estos algoritmos simulan los efectos de la lluvia, la evaporación y el transporte de sedimentos, lo que da lugar a terrenos con regeros fluviales, lagos y otras características precisas\cite{AlgoritmosHidrologia}.
    
    \item \textbf{Generación de Paisajes Multibioma}: AutoBiomes es un ejemplo de algoritmo de generación procedural que se enfoca en crear paisajes multibioma. Esta técnica combina enfoques sintéticos, basados en física y basados en ejemplos para generar terrenos realistas y visualmente diversos con múltiples biomas distintos\cite{GeneracionMultiBioma}.
    
    \item \textbf{Erosión Fluvial Basada en Grafos}: Un artículo reciente se propone un algoritmo de generación procedural de terrenos basado en una representación de grafo de erosión fluvial. Este algoritmo ofrece varias mejoras novedosas, incluyendo el uso de un mapa de restricción de altura con dos tipos de fuerzas de restricción localmente definidas, lo que da lugar a características de terreno más detalladas\cite{VillaValdes2015}.
    
\end{itemize}


\section{Antecedentes en Unity para la Generación Procedural de Terrenos}

Introducción:

Dada la irrupción de la generación de terreno procedural, en esta sección exploraremos las herramientas y recursos que existen en Unity para llevar a cabo esta técnica, así como algunas de las características clave de los assets disponibles en la Tienda de Assets de Unity.

\subsection{Herramientas y Recursos de Unity}

Unity proporciona varias herramientas y recursos para la generación procedural de terrenos. A continuación, se presentan algunas de las herramientas y recursos más populares:

\begin{itemize}
    \item \textbf{Terrain Engine de Unity}: El motor de terreno incorporado de Unity permite a los desarrolladores crear y modificar terrenos utilizando una variedad de herramientas, como pinceles, mapas de texturas y mapas de alturas. El motor de terreno puede utilizarse junto con técnicas de generación procedural para crear todo tipo de terrenos \cite{UnityTerrain}.
    
    \item \textbf{Comunidad}: Existen varios tutoriales, cursos, repositorios y recursos disponibles online creados por la basta comunidad de Unity que cubren la generación procedural, incluyendo la generación de terrenos. Estos recursos abordan temas como funciones de ruido, fractales y simulación física, y proporcionan instrucciones paso a paso para crear terrenos procedurales .
    
    \item \textbf{TerrainGenerator}: TerrainGenerator es una herramienta gratuita y de código abierto para Unity que permite a los desarrolladores crear terrenos procedurales utilizando algoritmos de ruido, simulación física y materiales personalizados. La herramienta puede crear una mesh de terreno, una mesh de agua y colocar objetos de forma aleatoria en una escena\cite{TerrainGenerator}.
    
    \item \textbf{Procedural Worlds}: Procedural Worlds es un conjunto de herramientas para Unity que permite a los desarrolladores crear y entregar contenido procedural. Las herramientas incluyen Gaia Pro, GeNa Pro y otras herramientas. \cite{ProceduralWorlds}.
    
\end{itemize}

\subsection{Plugins y Assets en Unity para la Generación de Terrenos Procedurales}

Existen varios plugins y assets en Unity que pueden ayudar en la generación de terrenos procedurales. Algunos de los más populares incluyen:

\begin{itemize}
    \item \textbf{Procedural Terrain Generator}: Este assets, disponible en la tienda de assets de Unity, permite a los desarrolladores crear terrenos procedurales utilizando una variedad de funciones de ruido, incluidas las funciones de Perlin y Simplex. También incluye características como simulación de erosión y mezcla de texturas \cite{ProceduralTerrainGenerator}.
    
    \item \textbf{Vista 2023 - Procedural Terrain Generator}: Este assets, también disponible en la tienda de assetss de Unity, permite a los desarrolladores crear terrenos procedurales utilizando una variedad de funciones de ruido, incluidas las funciones de Perlin, Simplex y Voronoi. También incluye características como simulación de erosión y mezcla de texturas \cite{Vista2023TerrainGenerator}.
    
    \item \textbf{MapMagic}: MapMagic es una herramienta de generación de terrenos que permite a los desarrolladores crear terrenos procedurales utilizando un sistema basado en nodos. Incluye características como simulación de erosión, mezcla de texturas y generación de biomas\cite{MapMagicTerrain}.
    
\end{itemize}

\subsection{Características del asset Procedural Terrain Generator en la tienda de assets de Unity}

El asset Procedural Terrain Generator en la tienda de assets de Unity, desarrollado por Nuance Studios, ofrece varias características para crear terrenos procedurales. Estas características incluyen:

\begin{itemize}
    \item \textbf{Generación Procedural}: El asset permite a los desarrolladores crear terrenos utilizando diversas funciones de ruido, incluyendo ruido de Perlin, ruido de Simplex y ruido de Voronoi. Esto posibilita la generación de paisajes realistas y diversos\cite{ProceduralTerrainGenerator}.
    
    \item \textbf{Personalización}: Los usuarios pueden personalizar fácilmente el terreno ajustando parámetros como escala, frecuencia y amplitud. Esta flexibilidad permite la creación de terrenos únicos y personalizados\cite{ProceduralTerrainGenerator}.
    
    \item \textbf{Simulación de Erosión}: El asset incluye una característica de simulación de erosión incorporada que se puede utilizar para crear lechos de ríos realistas, valles y otras características de terreno relacionadas con la erosión\cite{ProceduralTerrainGenerator}.
    
    \item \textbf{Mezcla de Texturas}: Los desarrolladores pueden mezclar múltiples texturas en el terreno, lo que permite la creación de paisajes más detallados y visualmente atrassets\cite{ProceduralTerrainGenerator}.
    
    \item \textbf{Compatibilidad}: El asset Procedural Terrain Generator es compatible con las versiones de Unity 5.3.4 o superiores \cite{ProceduralTerrainGenerator}.
\end{itemize}

Además del asset Procedural Terrain Generator, existen otros assets disponibles en la tienda de assets de Unity que ofrecen características similares para la generación de terrenos procedurales, como Vista 2023 - Procedural Terrain Generator de Pinwheel Studio\cite{Vista2023TerrainGenerator} y Tellus - Procedural Terrain Generator de Darkcom Dev\cite{TellusTerrainGenerator}. Estos assets pueden utilizarse para mejorar las capacidades de generación de terrenos de Unity y acelerar el proceso de desarrollo de juegos.
\newpage

\section{Aplicaciones de la Generación Procedural de Terreno}

\subsection{Aplicaciones en Videojuegos}

La generación procedural de terreno posee numerosas aplicaciones en la industria de los videojuegos. Algunas de las aplicaciones más destacadas incluyen:

\begin{itemize}
    \item \textbf{Aumento de la Rejugabilidad}: Se pueden crear variaciones infinitas del entorno de un juego gracias a la genración procedural, lo que aumenta la rejugabilidad\cite{Rejugabilidad}.

    \item \textbf{Ahorro de Tiempo en la Creación de Assets}: Al generar el terreno de forma procedural, los desarrolladores pueden ahorrar tiempo en la creación de assets y centrarse en otros aspectos del desarrollo del juego\cite{AhorroTiempo}.

    \item \textbf{Creación de Entornos Únicos}: Puede usarse para crear entornos únicos que serían difíciles o imposibles de crear manualmente\cite{EntornosUnicos}.

    \item \textbf{Generación de Jugabilidad Aleatoria}: La generación procedural de terreno puede utilizarse para crear jugabilidad aleatoria, como mapas, niveles, enemigos y armas aleatorias. Esto añade un elemento de imprevisibilidad y desafío al juego\cite{JugabilidadAleatoria}.

    \item \textbf{Creación de Nuevas Mecánicas de Juego}: También puede utilizarse como una mecánica de juego, como la creación de nuevos entornos para que el jugador explore o la generación de rompecabezas y desafíos\cite{NuevasMecanicas}.
\end{itemize} 

La generación procedural de terreno se ha convertido en una técnica cada vez más popular en el desarrollo de videojuegos, ofreciendo numerosos beneficios como el aumento de la rejugabilidad, el ahorro de tiempo y la creación de entornos únicos. A medida que la tecnología continúa avanzando, podemos esperar ver técnicas de generación de terreno aún más sofisticadas y realistas en los futuros videojuegos.

\subsection{Aplicaciones en Simulaciones Científicas}

La generación procedural de terrenos tiene diversas aplicaciones en simulaciones científicas, particularmente en los campos de la geología, hidrología y ciencias ambientales. Algunas de las aplicaciones destacadas incluyen:

\begin{enumerate}
    \item \textbf{Modelado Geológico}: Al simular procesos geológicos como la actividad tectónica y la erosión, los desarrolladores pueden crear terrenos que representen con precisión paisajes del mundo real\cite{GeologiaDocumento}.
    
    \item \textbf{Modelado Hidrológico}: Al simular el flujo de agua sobre el terreno, los desarrolladores pueden crear modelos que representen con precisión sistemas de agua del mundo real, como ríos, lagos y cuencas hidrográficas\cite{HidrologiaDocumento}.
    
    \item \textbf{Modelado Ambiental}: Al generar terrenos que representen con precisión entornos del mundo real, los desarrolladores pueden crear simulaciones más precisas y realistas\cite{AdaptiveTerrainGeneration}.
    
    \item \textbf{Compresión de Datos}: La generación procedural de terrenos se puede utilizar para comprimir grandes cantidades de datos de terreno en un archivo de menor tamaño. Al generar terrenos de manera procedural, los desarrolladores pueden crear terrenos sobre la marcha, reduciendo la necesidad de grandes cantidades de datos de terreno pregenerados\cite{RealWorldDataTerrain}.
\end{enumerate}
 
La generación procedural de terrenos tiene numerosas aplicaciones en simulaciones científicas, ofreciendo beneficios como mayor precisión, compresión de datos y la capacidad de simular sistemas naturales complejos.

\subsection{Aplicaciones en Realidad Virtual y Aumentada}

La generación procedural de terrenos tiene varias aplicaciones en la realidad virtual (RV) y la realidad aumentada (RA), ofreciendo beneficios como mayor inmersión, interactividad y realismo. Algunas de las aplicaciones destacadas incluyen:

\begin{itemize}
    
    \item \textbf{Generación interactiva de terrenos virtuales utilizando marcadores de RA}: La creación de terrenos virtuales utilizando marcadores de RA podría sería una apliación de la generación procedural en este ámbito. Esta técnica permite a los usuarios crear y modificar terrenos en tiempo real, brindando una experiencia más atractiva e interactiva\cite{ARInteractiveTerrain}.
    
    \item \textbf{Generación de RV procedural a partir de espacios físicos 3D reconstruidos}: La generación procedural de terrenos puede utilizarse para generar entornos de RV a partir de espacios físicos 3D reconstruidos. Por lo que los usuarios pueden explorar entornos del mundo real en RV de manera más inmersiva\cite{ARInteractiveTerrain}.
    
    \item \textbf{Generación procedural de escenas de RV}: La generación procedural de terrenos puede utilizarse para generar escenas de RV, como islas u otros entornos. Esta técnica permite a los desarrolladores crear entornos inmersivos que pueden explorarse en RV\cite{VRSceneGeneration}.
    
    \item \textbf{Experiencias de RV sobre la marcha mientras se camina dentro de grandes entornos de edificios del mundo real desconocidos}: La generación procedural de terrenos puede utilizarse para generar experiencias de RV mientras se camina dentro de grandes entornos de edificios del mundo real desconocidos, brindando una experiencia más inmersiva e interactiva\cite{ProceduralContentCreation}.
\end{itemize}

\subsection{Aplicaciones en Animación y Películas} 

La generación procedural de terrenos tiene muchas aplicaciones en la industria de la animación y el cine. Esto ofrece ventajas como una mayor eficiencia, flexibilidad y creatividad. Algunas de las aplicaciones son:

\begin{itemize}
    \item \textbf{Películas animadas generadas proceduralmente}: Esta técnica se utiliza para crear películas animadas con entornos y personajes generados de manera aleatoria. Esto permite crear películas visualmente únicas e interesantes con un esfuerzo mínimo en la creación manual\cite{ProcedurallyGeneratedAnimatedFilms}.
    
    \item \textbf{Generación procedural de objetos 3D y animaciones}: La generación procedural se emplea para crear objetos 3D y animaciones para películas, como los diseños de personajes, animaciones y diálogos de personajes no jugadores. Esto da como resultado contenido diverso e interesante con poco esfuerzo manual\cite{ProceduralGeneration3DObjects}.
    
    \item \textbf{Creación rápida de escenarios precisos}: En la industria cinematográfica, la generación procedural se usa para crear escenarios visualmente precisos de manera rápida, habilitando la creación de entornos que se pueden explorar en películas\cite{CreatingVisuallyInterestingSpaces}.
    
    \item \textbf{Creación de cortos animados con generación procedural}: La generación procedural de terrenos también se aplica para crear cortos animados mediante la generación de formas y campos de distancia por código. Esto reduce la complejidad de los archivos y facilita la producción de películas con un tamaño de archivo mínimo\cite{ShortAnimatedMoviesProceduralGeneration}.
\end{itemize}

\section{Desafíos y Tendencias Futuras}

La generación procedural de terrenos presenta una serie de desafíos y tendencias que son esenciales para su evolución y aplicación en el desarrollo de juegos y otras áreas. Algunos de estos desafíos y tendencias incluyen:

\begin{enumerate}
    \item \textbf{Consistencia y Coherencia}: Garantizar que el terreno generado sea consistente y coherente en diversas plataformas y dispositivos puede ser un desafío. Para abordar esto, se utilizan técnicas como las funciones de ruido, la síntesis de terreno y la generación de contenido procedural (PCG)\cite{ConsistencyCoherence}.
    
    \item \textbf{Equilibrio en la Jugabilidad}: Crear un mundo generado que sea justo y mantenga un nivel adecuado de desafío puede ser complicado. Encontrar el equilibrio entre la dificultad del terreno y los elementos de juego es esencial para proporcionar una experiencia satisfactoria para los jugadores\cite{BalancingGameplay}.
    
    \item \textbf{Realismo y Variedad}: Lograr un equilibrio entre un terreno realista y paisajes diversos y visualmente atractivos es un reto. Las tendencias futuras se centran en mejorar la apariencia realista y la variedad de los terrenos generados mediante algoritmos y técnicas más avanzadas\cite{RealismVariety}.
    
    \item \textbf{Optimización de Rendimiento}: Generar terrenos complejos en tiempo real puede ser intensivo en recursos computacionales. Las tendencias futuras incluyen la optimización de los algoritmos de generación procedural de terrenos para mejorar el rendimiento y reducir el uso de recursos\cite{PerformanceOptimization}.
    
    \item \textbf{Integración con Otros Sistemas de Juego}: La generación procedural de terrenos debe integrarse de manera efectiva con otros sistemas de juego, como la simulación de física, la inteligencia artificial y el diseño de niveles. Las tendencias futuras implican el desarrollo de técnicas avanzadas para lograr esta integración sin problemas\cite{IntegrationWithGameSystems}.
\end{enumerate}

En resumen, la generación procedural de terrenos es una tendencia en crecimiento en el desarrollo de juegos y otras áreas, lo que brinda la oportunidad de crear mundos de juego más grandes, dinámicos e interesantes. Abordar los desafíos y adoptar las tendencias futuras en este campo conducirá a técnicas de generación de terrenos más avanzadas y realistas.


\chapter{Desarrollo de la solucion}
% % Contenidos del capítulo.
% Las secciones presentadas son orientativas y no representan
% necesariamente la organización que debe tener este capítulo.


% \subsubsection{2. Desarrollo de la Interfaz de Usuario}

% La interfaz de usuario desempeña un papel crucial en la usabilidad de la herramienta. Los objetivos relacionados con la interfaz de usuario incluyen:

% \begin{itemize}
%     \item Diseñar una interfaz de usuario intuitiva y fácil de usar para permitir a los usuarios configurar los parámetros de generación.
%     \item Implementar una interfaz de usuario que refleje de manera efectiva las opciones disponibles para la generación de terrenos.
% \end{itemize}

\section{Análisis}

\subsection{Objetivos de Implementación}

Los objetivos de implementación se centran en las metas técnicas y funcionales que se buscan alcanzar en el desarrollo de la herramienta de generación procedural de terrenos en Unity. Estos objetivos se dividen en los siguientes aspectos clave:

\subsubsection{1. Diseño de Algoritmos de Generación}

El objetivo principal en esta fase de implementación es diseñar algoritmos de generación de terrenos que sean eficientes y capaces de producir resultados convincentes. Esto incluye:

\begin{itemize}
    \item Investigar y seleccionar algoritmos de generación de terrenos adecuados para el proyecto.
    \item Diseñar algoritmos que permitan la creación de terrenos realistas y variados.
\end{itemize}

\subsubsection{2. Optimización del Rendimiento}

Para garantizar que la herramienta funcione de manera eficiente en diversas plataformas y escenarios de desarrollo, se establecen los siguientes objetivos:

\begin{itemize}
    \item Optimizar el rendimiento de los algoritmos de generación para maximizar los fotogramas por segundo.
    \item Implementar estrategias de cálculo paralelo utilizando el sistema de trabajos (Job System) de Unity para acelerar la generación de terrenos.
\end{itemize}

\subsubsection{3. Pruebas y Validación}

La validación de la herramienta es fundamental para garantizar su funcionamiento correcto. Los objetivos relacionados con las pruebas son:

\begin{itemize}
    \item Detectar posibles errores y problemas de rendimiento. Como la consistencia entre chunks vecinos dando lugar a terrenos conitnuos y la generación del terreno continua sin bajadas de fps notables.
    \item Validar que la herramienta genera terrenos realistas acorde a los parámetros con los que se configura.
    \item Comprobar que los resultados son visualmente integrables en juegos o proyectos desarrollados en Unity y que permite crear un terreno explorable.
\end{itemize}

\subsubsection{4. Mejoras en Realismo y Diferenciación de Alturas}

Además de los objetivos anteriores, se busca mejorar el realismo de los terrenos generados mediante la implementación de algoritmos de erosión. Los objetivos adicionales incluyen:

\begin{itemize}
    \item Investigar y aplicar algoritmos de erosión para simular procesos geológicos en los terrenos generados.
    \item Evaluar cómo los algoritmos de erosión mejoran la apariencia y autenticidad de los terrenos.
    \item Implementar la diferenciación de alturas en los terrenos mediante la asignación de colores según la elevación para una representación visual más rica y comprensible.
\end{itemize}

\subsection{Requisitos del Sistema}

\subsubsection{Requisitos Funcionales}

\begin{enumerate}
    \item \textbf{Generación de Terrenos:} El sistema debe ser capaz de generar terrenos de manera procedural en tiempo real, permitiendo a los usuarios especificar parámetros como tamaño, altura, erosión y demás configuraciones de terreno.
    
    \item \textbf{Continuidad del terreno:} El terreno no debe presentar discontinuidades y debe generar extensiones de terreno sin que haya una transción notable de uno a otro.
    
    \item \textbf{Algoritmos de Generación Configurables:} Los algoritmos de generación utilizados deben ser configurables, lo que permitirá a los usuarios ajustar los detalles de la generación según sus necesidades.
    
    \item \textbf{Optimización del Rendimiento:} El sistema debe estar optimizado para garantizar que la generación de terrenos sea eficiente en términos de uso de recursos y tiempos de carga.
    
    \item \textbf{Erosión Simulada:} Se deben implementar algoritmos de erosión para simular procesos geológicos y mejorar la apariencia de los terrenos generados.
    
    \item \textbf{Diferenciación de Alturas:} El sistema debe asignar colores a diferentes elevaciones del terreno para facilitar la visualización y comprensión de las características del terreno.
\end{enumerate}

\subsubsection{Requisitos No Funcionales}

\begin{enumerate}
    \item \textbf{Rendimiento:} El sistema debe ser capaz de generar terrenos en tiempo real sin experimentar retrasos notables en la ejecución.
    
    \item \textbf{Compatibilidad con Unity:} La herramienta debe integrarse perfectamente con el motor Unity, aprovechando sus capacidades y recursos.
    
    \item \textbf{Portabilidad:} El sistema debe ser compatible con múltiples plataformas y versiones de Unity, lo que permite a los desarrolladores utilizarlo en diversos proyectos.
    
    \item \textbf{Usabilidad:} El sistema debe ser intuititvo, con parámetros nombrados de manera que no cause confución en el usurio y expresen de manera clara su función.
    
    \item \textbf{Realismo Visual:} El sistema debe ser capaz de generar terrenos con realismo visual, incluyendo detalles naturales como montañas, valles.
    
    \item \textbf{Diferenciación de Alturas:} La herramienta debe permitir la diferenciación de alturas en el terreno mediante la asignación de colores o texturas específicas para representar diferentes elevaciones, facilitando la visualización y comprensión del terreno generado.

\end{enumerate}

\subsection{Arquitectura del Sistema}

\subsubsection{Visión General}
La arquitectura del sistema se basa en un enfoque modular que consta de varios componentes interconectados. El sistema se ha diseñado para ser altamente flexible y escalable, permitiendo la generación procedural de terrenos de manera eficiente. La arquitectura se centra en la generación de terrenos y su visualización en tiempo real.

\subsubsection{Componentes del Sistema}
Los componentes clave del sistema incluyen:

\begin{itemize}
    \item \textbf{MapGenerator:} Responsable de crear el terreno proceduralmente. Utiliza algoritmos de generación de ruido y permite la visualización en el editor de Unity.
     
    \item \textbf{MapDataGeneratorJob:} Se encarga de la generación de datos del terreno, como el mapa de altura y el mapa de colores, utilizando el Unity Job System.

    \item \textbf{Noise:} Contiene métodos para generar ruido Perlin, Simplex y Voronoi, que se utilizan en la generación del terreno.
            
    \item \textbf{ErosionJob:} Responsable de tratar cada índice del mapa de alturas generado con MapGenerator mediante  algoritmos de erosión para mejorar la apariencia del terreno, creando características como ríos y cañones.
    
    \item \textbf{MeshDataGeneratorJob:} Genera los datos de la malla del terreno, permitiendo la creación de mallas detalladas y eficientes en cuanto a rendimiento.
    
    \item \textbf{EndlessTerrain:} Controla la generación continua y la representación de terrenos, generando nuevos trozos de terreno y manejando la gestión de los mismos para lograr un mundo de juego sin fin.
    
    \item \textbf{TextureGenerator:} Se encarga de generar las texturas del terreno en función del mapa de altura y el gradiente de colores.

    \item \textbf{ConfigSettings:} Almacena las configuraciones del sistema, incluyendo parámetros de generación de terrenos, configuraciones de malla y opciones de visualización.
\end{itemize}

Esta arquitectura modular y bien definida permite una generación procedural de terrenos flexible y eficaz en Unity.

\subsubsection{Diagramas de Arquitectura}
A continuación, se presentan diagramas de arquitectura que muestran la estructura y las relaciones entre los componentes del sistema:

\subsubsection{Diagrama de Casos de Uso}

Para comprender mejor las interacciones entre los usuarios y el sistema, se han creado diagramas de casos de uso. Estos diagramas describen cómo los usuarios interactúan con el sistema y qué funcionalidades están disponibles para ellos.

\begin{figure}[H]
    \centering
    \includegraphics[width=0.5\textwidth]{img/UseCases.png}
    \caption{Diagrama de Casos de Uso del Proyecto.}
\end{figure}


El diagrama de secuencia detallará las interacciones entre los componentes del sistema, incluidos el \texttt{MapGenerator}, el \texttt{MeshGenerator}, el \texttt{EndlessTerrain}, y otros, durante la generación y visualización del terreno.

\begin{figure}[H]
    \centering
    \includegraphics[width=1\textwidth]{img/Diagrama de secuencia.png}
    \caption{Diagrama de Secuencia de Generación de Terreno.}
\end{figure}


\subsection{Tecnologías y Herramientas Utilizadas}

\subsubsection{Elección de Tecnologías}
La elección de las tecnologías específicas para este proyecto se basó en los siguientes criterios:

\begin{itemize}
    \item \textbf{Unity 3D:} Se eligió Unity como motor de desarrollo debido a su versatilidad y capacidad para crear aplicaciones interactivas en 3D. Unity proporciona una amplia gama de herramientas y recursos que facilitan el desarrollo de juegos y simulaciones.
    
    
    \item \textbf{Unity Job System:} Para optimizar el rendimiento en la generación de terrenos, se utiliza el Unity Job System, que permite la paralelización de tareas en múltiples núcleos de CPU.
    
    \item \textbf{Burst Compiler:} La herramienta Burst Compiler se utiliza para compilar el código C \# en código nativo altamente optimizado, mejorando aún más el rendimiento de la generación de terrenos.
    
    \item \textbf{Perlin Noise y Simplex Noise:} Se implementan algoritmos de ruido Perlin y Simplex para la generación de terrenos. Estos algoritmos proporcionan resultados realistas y variados.
    
    \item \textbf{Herramientas de Diseño 3D:} Se utilizan herramientas de diseño 3D, como Blender y Substance Painter, para crear modelos y texturas que se aplicarán al terreno.
\end{itemize}

Estas tecnologías se eligieron cuidadosamente para garantizar un rendimiento óptimo y una alta calidad en la generación de terrenos en tiempo real.

\subsubsection{Lenguajes de Programación}
\begin{itemize}
    \item \textbf{C\#:} Este proyecto está programado enteramente utilizando C\# como lenguaje de programación. C\# es el lenguaje que utilizan los componentes Scripts de Unity por lo que está altamente integrado con el motor, y es un lenguaje orientado a objetos que ofrece un alto rendimiento y facilidad de uso.
    
\end{itemize}

\subsection{Herramientas de Desarrollo}

Durante el desarrollo de este proyecto, se utilizaron diversas herramientas y software que desempeñaron un papel fundamental en la planificación, implementación y gestión del trabajo. A continuación, se detallan las principales herramientas de desarrollo utilizadas:

\begin{itemize}
    \item \textbf{IDE Principal:} JetBrains Rider fue el IDE principal utilizado para el desarrollo en C\#. Rider proporcionó un entorno de desarrollo integrado eficiente y potente para la escritura de código, depuración y pruebas del proyecto.
    
    \item \textbf{Diagrama de Clases (Visual Studio):} Visual Studio se utilizó específicamente para la creación de diagramas de clases, lo que permitió una representación visual clara de la estructura del proyecto y las relaciones entre las clases.
    
    \item \textbf{Diagramas (VS Code con el Plugin draw.io):} Visual Studio Code, junto con el plugin draw.io, se utilizó para crear varios tipos de diagramas, incluidos los diagramas de casos de uso y diagramas de actividad. Estas representaciones gráficas ayudaron a comprender y comunicar el flujo de trabajo del sistema.
    
    \item \textbf{Memoria en LaTeX:} La documentación y memoria del proyecto se crearon utilizando LaTeX, con el editor VS Code.
    
    \item \textbf{Control de Versiones (Git y GitHub):} Git se utilizó para el control de versiones del código fuente del proyecto. GitHub se empleó como plataforma de alojamiento para el repositorio de Git, lo que facilitó la colaboración en equipo y el seguimiento de cambios.
    
    \item \textbf{Gestión de Tareas (Trello):} Trello se utilizó para la gestión de tareas y la planificación del proyecto. La herramienta permitió organizar y priorizar tareas, así como realizar un seguimiento del progreso de cada elemento del proyecto.
    
    \item \textbf{Burst Compiler:} El Burst Compiler se utilizó para compilar el código C\# en código nativo altamente optimizado, lo que contribuyó significativamente a mejorar el rendimiento en la generación de terrenos.
\end{itemize}

Estas herramientas desempeñaron un papel esencial en la realización exitosa del proyecto, proporcionando las capacidades necesarias para el desarrollo, la documentación, la colaboración en equipo y la optimización de rendimiento.

\subsection{Desafíos y Decisiones de Diseño}

\subsubsection{Desafíos Técnicos}
Durante el diseño y desarrollo del proyecto, se enfrentaron varios desafíos técnicos significativos. Algunos de los desafíos más destacados incluyeron:

\begin{itemize}
    \item \textbf{Optimización de Rendimiento:} Lograr un rendimiento óptimo en la generación procedural de terrenos en tiempo real fue uno de los principales desafíos técnicos. Se implementó el Unity Job System y el Burst Compiler para abordar este desafío.
    
    \item \textbf{Generación Realista:} La generación de terrenos realistas y variados implicó la implementación de algoritmos de ruido Perlin, Simplex y Voronoi, así como la configuración adecuada de parámetros como escalas y octavas.
    
    \item \textbf{Erosión y Características Naturales:} Incorporar algoritmos de erosión para simular características naturales como ríos y cañones fue un desafío adicional.
    
    \item \textbf{Continuidad del terreno:} La continuidad del terreno nuevo que se genera, sin notar la transición entre partes de terreno generadas e integradas al terreno que ya había supuso otro tema técnico que hubo que resolver.
    
    \item \textbf{Corrección de borde en erosión:} Dado que a eorisón se reliza teneindo en cuenta las alturas de cada fragmetnosde terreno y equilibrándolas, porduce incosistencias con las partes de terreno. Resolver esto fue otro desafío.
\end{itemize}

\subsubsection{Decisiones de Diseño}
Las decisiones de diseño desempeñaron un papel fundamental en la arquitectura y funcionalidad del proyecto. Algunas de las decisiones clave incluyeron:

\begin{itemize}
    \item \textbf{Uso del Unity Job System:} Se decidió utilizar el Unity Job System para la paralelización de tareas y optimizar la generación de terrenos, lo que resultó en un rendimiento mejorado.
    
    \item \textbf{Selección de Algoritmos de Ruido:} La elección de implementar algoritmos de ruido Perlin y Simplex permitió generar terrenos realistas y variados con una apariencia natural.
    
    \item \textbf{Erosión para Características Naturales:} La incorporación de algoritmos de erosión en el diseño permitió crear características naturales como ríos y cañones, mejorando la apariencia general del terreno.
    
    \item \textbf{Elección de nave como explorador} La elección de elección de una nave como explorador de terreno se debió a que las irregularidades del terreno para terrenos escarpados podrían complicar la exploración, además de que con un sobrevuelo se pdoría ver mejor el rendimiento de la generació.
\end{itemize}

\subsubsection{Alternativas Consideradas}
Antes de tomar las decisiones de diseño finales, se consideraron varias alternativas, incluyendo:

\begin{itemize}
    \item \textbf{Otras Tecnologías de Generación de Terrenos:} Se evaluaron diferentes tecnologías y enfoques para la generación de terrenos, como el uso de mapas de altura pregenerados versus generación procedural en tiempo real.
    
    \item \textbf{Métodos de Optimización:} Se exploraron diversas técnicas de optimización además del Unity Job System, como el uso de GPU para cálculos intensivos o la paralelización mediante threads manual.
    
    \item \textbf{Otros Algoritmos de Ruido:} Se investigaron algoritmos de ruido alternativos además de Perlin y Simplex para determinar cuáles producirían los resultados deseados y técnicas de generación, como el algoritmo diamante-cuadrado. Pero se optó por el ruido debido a que facilitaba la consistencia entre lso bordes de partes de terreno nuevas generadas
\end{itemize}

\subsection{Planificación del Desarrollo}

\subsubsection{Metodología de Desarrollo}
El proyecto siguió una metodología de desarrollo ágil, lo que permitió una adaptación flexible a medida que se abordaban desafíos técnicos y se tomaban decisiones de diseño. Se realizaron reuniones periódicas de revisión y planificación para ajustar el enfoque según fuera necesario. Cada día se fueron subiendo incrementos de desarrollo al repositorio remoto, gestioando un control de cuáles habían sido las mejroas subidas, cuál era el estado del proyecto y cuáles debían ser los siguientes objetivos.

\subsubsection{Gestión de Tiempo}
La gestión del tiempo se realizó mediante una planificación detallada en Trello. Creando pilas de tareas "por hacer", "en desarrollo", "termiandas" y "mejorables".

\subsubsection{Recursos Necesarios}
Los recursos necesarios para llevar a cabo el desarrollo incluyeron:

\begin{itemize}
    \item Personal de Desarrollo: Para este proyecto el personal fue una única persona que ocupó todos los roles del desarrollo y un product owner que especificaba los requisitos que debía cumplir el proyecto.
    
    \item Hardware: Se utilizó un equipo portátil con procesador i5-11400H con 2.7GHz, 16 GB de RAM, 1TB de memoria en disco y tarjeta gráfica NVidia 3060 con 6GB de RAM. El equipo contaba con windows 10 Home como sistema operativo. 
    
    \item Software: Se requirieron herramientas como Unity3D, JetBrains Rider, Visual Studio, VS Code con el plugin draw.io, LaTeX y el Burst Compiler para el desarrollo y la documentación del proyecto.
\end{itemize}

La gestión eficaz de estos recursos fue fundamental para el éxito del proyecto.


\section{Diseño}

\subsection{Diseño de la Arquitectura}

\subsubsection{Diseño de la Arquitectura del Software}
Describe la arquitectura de software que has diseñado, incluyendo patrones arquitectónicos si los has aplicado.

\subsubsection{Diseño de la Interfaz de Usuario (UI)}
Explica cómo has diseñado la interfaz de usuario, incluyendo esquemas de diseño y experiencia de usuario (UX).

\subsubsection{Diagramas de Clases}
Incluye diagramas de clases que representen la estructura de clases de tu proyecto y sus relaciones.

\subsection{Diseño Detallado}

\subsubsection{Diagramas de Secuencia}
Muestra los diagramas de secuencia que ilustran la interacción entre los componentes clave de tu sistema.

\subsubsection{Diagramas de Actividad}
Incluye diagramas de actividad que representen los flujos de trabajo y procesos en tu aplicación.

\subsubsection{Diagramas de Estado}
Explica los estados y transiciones de los elementos que tienen comportamientos específicos.

\subsubsection{Diseño de Base de Datos}
Si tu proyecto incluye una base de datos, describe su diseño, esquema y relaciones.

\subsection{Decisiones de Diseño}

\subsubsection{Decisiones de Diseño Clave}
Destaca las decisiones de diseño más importantes que afectaron la arquitectura y funcionalidad de tu proyecto.

\subsubsection{Consideraciones de Rendimiento}
Explica las consideraciones de rendimiento que tuviste en cuenta en el diseño.

\subsubsection{Seguridad y Privacidad}
Describe cómo abordaste la seguridad y la privacidad en el diseño de tu aplicación.

\subsection{Prototipado}

\subsubsection{Prototipos de Interfaz}
Muestra prototipos de la interfaz de usuario para visualizar el diseño antes de la implementación.

\subsubsection{Pruebas de Concepto}
Si realizaste pruebas de concepto, explica los resultados y cómo influyeron en el diseño final.

\subsection{Planificación de Desarrollo}

\subsubsection{Cronograma de Desarrollo}
Incluye el cronograma detallado de desarrollo, indicando hitos y fechas clave.

\subsubsection{Asignación de Recursos}
Describe cómo asignaste recursos, incluyendo personal y hardware, para llevar a cabo el diseño y desarrollo.

\subsubsection{Plan de Pruebas}
Explica tu plan de pruebas de diseño y cómo verificarás que el diseño cumple con los requisitos.

\subsubsection{Control de Cambios}
Detalla cómo gestionarás los cambios en el diseño a medida que evolucione el proyecto.

\subsection{Consideraciones Éticas y Legales}

\subsubsection{Ética en el Diseño}
Aborda cualquier consideración ética relacionada con el diseño de tu proyecto.

\subsubsection{Consideraciones Legales}
Describe cualquier consideración legal relacionada con el diseño, como licencias de software y derechos de autor.



\section{Implementación}

\subsection{Entorno de Desarrollo}

\subsubsection{Herramientas de Desarrollo}
Enumera las herramientas de desarrollo que has utilizado, incluyendo el entorno de desarrollo integrado (IDE), editores de código, y otros programas relevantes.

\subsubsection{Lenguajes de Programación}
Describe los lenguajes de programación que has empleado en tu proyecto y su papel en la implementación.

\subsubsection{Frameworks y Bibliotecas}
Menciona cualquier framework o biblioteca de terceros que hayas utilizado y cómo contribuyeron a la implementación.

\subsection{Estructura del Código Fuente}

\subsubsection{Organización de Directorios}
Explica la estructura de directorios de tu proyecto y cómo organizaste los archivos de código fuente.

\subsubsection{Módulos y Componentes Principales}
Identifica los módulos y componentes principales de tu aplicación y cómo se relacionan entre sí.

\subsubsection{Flujo de Datos}
Describe cómo fluyen los datos a través de tu aplicación, desde la entrada hasta la salida.

\subsection{Algoritmos y Técnicas}

\subsubsection{Descripción de Algoritmos Clave}
Explica los algoritmos clave que implementaste en tu proyecto y cómo contribuyen a su funcionamiento.

\subsubsection{Técnicas de Optimización}
Si aplicaste técnicas de optimización, como paralelización o caching, descríbelas y su impacto en el rendimiento.

\subsection{Desarrollo de Características}

\subsubsection{Implementación de Funcionalidades}
Detalla la implementación de las principales funcionalidades de tu proyecto, destacando aspectos relevantes.

\subsubsection{Integración de Componentes}
Explica cómo integras los diferentes componentes y módulos en tu aplicación.

\subsection{Pruebas y Depuración}

\subsubsection{Pruebas Unitarias}
Describe las pruebas unitarias que realizaste y cómo ayudaron a identificar y corregir errores.

\subsubsection{Pruebas de Integración}
Explica cómo llevaste a cabo las pruebas de integración y qué desafíos enfrentaste.

\subsubsection{Depuración y Solución de Problemas}
Detalla el proceso de depuración que seguiste para identificar y resolver problemas en el código.




% \chapter{Análisis}
% Contenidos del capítulo.
% Las secciones presentadas son orientativas y no representan
% necesariamente la organización que debe tener este capítulo.


% \subsubsection{2. Desarrollo de la Interfaz de Usuario}

% La interfaz de usuario desempeña un papel crucial en la usabilidad de la herramienta. Los objetivos relacionados con la interfaz de usuario incluyen:

% \begin{itemize}
%     \item Diseñar una interfaz de usuario intuitiva y fácil de usar para permitir a los usuarios configurar los parámetros de generación.
%     \item Implementar una interfaz de usuario que refleje de manera efectiva las opciones disponibles para la generación de terrenos.
% \end{itemize}

\section{Análisis}

\subsection{Objetivos de Implementación}

Los objetivos de implementación se centran en las metas técnicas y funcionales que se buscan alcanzar en el desarrollo de la herramienta de generación procedural de terrenos en Unity. Estos objetivos se dividen en los siguientes aspectos clave:

\subsubsection{1. Diseño de Algoritmos de Generación}

El objetivo principal en esta fase de implementación es diseñar algoritmos de generación de terrenos que sean eficientes y capaces de producir resultados convincentes. Esto incluye:

\begin{itemize}
    \item Investigar y seleccionar algoritmos de generación de terrenos adecuados para el proyecto.
    \item Diseñar algoritmos que permitan la creación de terrenos realistas y variados.
\end{itemize}

\subsubsection{2. Optimización del Rendimiento}

Para garantizar que la herramienta funcione de manera eficiente en diversas plataformas y escenarios de desarrollo, se establecen los siguientes objetivos:

\begin{itemize}
    \item Optimizar el rendimiento de los algoritmos de generación para maximizar los fotogramas por segundo.
    \item Implementar estrategias de cálculo paralelo utilizando el sistema de trabajos (Job System) de Unity para acelerar la generación de terrenos.
\end{itemize}

\subsubsection{3. Pruebas y Validación}

La validación de la herramienta es fundamental para garantizar su funcionamiento correcto. Los objetivos relacionados con las pruebas son:

\begin{itemize}
    \item Detectar posibles errores y problemas de rendimiento. Como la consistencia entre chunks vecinos dando lugar a terrenos conitnuos y la generación del terreno continua sin bajadas de fps notables.
    \item Validar que la herramienta genera terrenos realistas acorde a los parámetros con los que se configura.
    \item Comprobar que los resultados son visualmente integrables en juegos o proyectos desarrollados en Unity y que permite crear un terreno explorable.
\end{itemize}

\subsubsection{4. Mejoras en Realismo y Diferenciación de Alturas}

Además de los objetivos anteriores, se busca mejorar el realismo de los terrenos generados mediante la implementación de algoritmos de erosión. Los objetivos adicionales incluyen:

\begin{itemize}
    \item Investigar y aplicar algoritmos de erosión para simular procesos geológicos en los terrenos generados.
    \item Evaluar cómo los algoritmos de erosión mejoran la apariencia y autenticidad de los terrenos.
    \item Implementar la diferenciación de alturas en los terrenos mediante la asignación de colores según la elevación para una representación visual más rica y comprensible.
\end{itemize}

\subsection{Requisitos del Sistema}

\subsubsection{Requisitos Funcionales}

\begin{enumerate}
    \item \textbf{Generación de Terrenos:} El sistema debe ser capaz de generar terrenos de manera procedural en tiempo real, permitiendo a los usuarios especificar parámetros como tamaño, altura, erosión y demás configuraciones de terreno.
    
    \item \textbf{Continuidad del terreno:} El terreno no debe presentar discontinuidades y debe generar extensiones de terreno sin que haya una transción notable de uno a otro.
    
    \item \textbf{Algoritmos de Generación Configurables:} Los algoritmos de generación utilizados deben ser configurables, lo que permitirá a los usuarios ajustar los detalles de la generación según sus necesidades.
    
    \item \textbf{Optimización del Rendimiento:} El sistema debe estar optimizado para garantizar que la generación de terrenos sea eficiente en términos de uso de recursos y tiempos de carga.
    
    \item \textbf{Erosión Simulada:} Se deben implementar algoritmos de erosión para simular procesos geológicos y mejorar la apariencia de los terrenos generados.
    
    \item \textbf{Diferenciación de Alturas:} El sistema debe asignar colores a diferentes elevaciones del terreno para facilitar la visualización y comprensión de las características del terreno.
\end{enumerate}

\subsubsection{Requisitos No Funcionales}

\begin{enumerate}
    \item \textbf{Rendimiento:} El sistema debe ser capaz de generar terrenos en tiempo real sin experimentar retrasos notables en la ejecución.
    
    \item \textbf{Compatibilidad con Unity:} La herramienta debe integrarse perfectamente con el motor Unity, aprovechando sus capacidades y recursos.
    
    \item \textbf{Portabilidad:} El sistema debe ser compatible con múltiples plataformas y versiones de Unity, lo que permite a los desarrolladores utilizarlo en diversos proyectos.
    
    \item \textbf{Usabilidad:} El sistema debe ser intuititvo, con parámetros nombrados de manera que no cause confución en el usurio y expresen de manera clara su función.
    
    \item \textbf{Realismo Visual:} El sistema debe ser capaz de generar terrenos con realismo visual, incluyendo detalles naturales como montañas, valles.
    
    \item \textbf{Diferenciación de Alturas:} La herramienta debe permitir la diferenciación de alturas en el terreno mediante la asignación de colores o texturas específicas para representar diferentes elevaciones, facilitando la visualización y comprensión del terreno generado.

\end{enumerate}

\subsection{Arquitectura del Sistema}

\subsubsection{Visión General}
La arquitectura del sistema se basa en un enfoque modular que consta de varios componentes interconectados. El sistema se ha diseñado para ser altamente flexible y escalable, permitiendo la generación procedural de terrenos de manera eficiente. La arquitectura se centra en la generación de terrenos y su visualización en tiempo real.

\subsubsection{Componentes del Sistema}
Los componentes clave del sistema incluyen:

\begin{itemize}
    \item \textbf{MapGenerator:} Responsable de crear el terreno proceduralmente. Utiliza algoritmos de generación de ruido y permite la visualización en el editor de Unity.
     
    \item \textbf{MapDataGeneratorJob:} Se encarga de la generación de datos del terreno, como el mapa de altura y el mapa de colores, utilizando el Unity Job System.

    \item \textbf{Noise:} Contiene métodos para generar ruido Perlin, Simplex y Voronoi, que se utilizan en la generación del terreno.
            
    \item \textbf{ErosionJob:} Responsable de tratar cada índice del mapa de alturas generado con MapGenerator mediante  algoritmos de erosión para mejorar la apariencia del terreno, creando características como ríos y cañones.
    
    \item \textbf{MeshDataGeneratorJob:} Genera los datos de la malla del terreno, permitiendo la creación de mallas detalladas y eficientes en cuanto a rendimiento.
    
    \item \textbf{EndlessTerrain:} Controla la generación continua y la representación de terrenos, generando nuevos trozos de terreno y manejando la gestión de los mismos para lograr un mundo de juego sin fin.
    
    \item \textbf{TextureGenerator:} Se encarga de generar las texturas del terreno en función del mapa de altura y el gradiente de colores.

    \item \textbf{ConfigSettings:} Almacena las configuraciones del sistema, incluyendo parámetros de generación de terrenos, configuraciones de malla y opciones de visualización.
\end{itemize}

Esta arquitectura modular y bien definida permite una generación procedural de terrenos flexible y eficaz en Unity.

\subsubsection{Diagramas de Arquitectura}
A continuación, se presentan diagramas de arquitectura que muestran la estructura y las relaciones entre los componentes del sistema:

\subsubsection{Diagrama de Casos de Uso}

Para comprender mejor las interacciones entre los usuarios y el sistema, se han creado diagramas de casos de uso. Estos diagramas describen cómo los usuarios interactúan con el sistema y qué funcionalidades están disponibles para ellos.

\begin{figure}[H]
    \centering
    \includegraphics[width=0.5\textwidth]{img/UseCases.png}
    \caption{Diagrama de Casos de Uso del Proyecto.}
\end{figure}


El diagrama de secuencia detallará las interacciones entre los componentes del sistema, incluidos el \texttt{MapGenerator}, el \texttt{MeshGenerator}, el \texttt{EndlessTerrain}, y otros, durante la generación y visualización del terreno.

\begin{figure}[H]
    \centering
    \includegraphics[width=1\textwidth]{img/Diagrama de secuencia.png}
    \caption{Diagrama de Secuencia de Generación de Terreno.}
\end{figure}


\subsection{Tecnologías y Herramientas Utilizadas}

\subsubsection{Elección de Tecnologías}
La elección de las tecnologías específicas para este proyecto se basó en los siguientes criterios:

\begin{itemize}
    \item \textbf{Unity 3D:} Se eligió Unity como motor de desarrollo debido a su versatilidad y capacidad para crear aplicaciones interactivas en 3D. Unity proporciona una amplia gama de herramientas y recursos que facilitan el desarrollo de juegos y simulaciones.
    
    
    \item \textbf{Unity Job System:} Para optimizar el rendimiento en la generación de terrenos, se utiliza el Unity Job System, que permite la paralelización de tareas en múltiples núcleos de CPU.
    
    \item \textbf{Burst Compiler:} La herramienta Burst Compiler se utiliza para compilar el código C \# en código nativo altamente optimizado, mejorando aún más el rendimiento de la generación de terrenos.
    
    \item \textbf{Perlin Noise y Simplex Noise:} Se implementan algoritmos de ruido Perlin y Simplex para la generación de terrenos. Estos algoritmos proporcionan resultados realistas y variados.
    
    \item \textbf{Herramientas de Diseño 3D:} Se utilizan herramientas de diseño 3D, como Blender y Substance Painter, para crear modelos y texturas que se aplicarán al terreno.
\end{itemize}

Estas tecnologías se eligieron cuidadosamente para garantizar un rendimiento óptimo y una alta calidad en la generación de terrenos en tiempo real.

\subsubsection{Lenguajes de Programación}
\begin{itemize}
    \item \textbf{C\#:} Este proyecto está programado enteramente utilizando C\# como lenguaje de programación. C\# es el lenguaje que utilizan los componentes Scripts de Unity por lo que está altamente integrado con el motor, y es un lenguaje orientado a objetos que ofrece un alto rendimiento y facilidad de uso.
    
\end{itemize}

\subsection{Herramientas de Desarrollo}

Durante el desarrollo de este proyecto, se utilizaron diversas herramientas y software que desempeñaron un papel fundamental en la planificación, implementación y gestión del trabajo. A continuación, se detallan las principales herramientas de desarrollo utilizadas:

\begin{itemize}
    \item \textbf{IDE Principal:} JetBrains Rider fue el IDE principal utilizado para el desarrollo en C\#. Rider proporcionó un entorno de desarrollo integrado eficiente y potente para la escritura de código, depuración y pruebas del proyecto.
    
    \item \textbf{Diagrama de Clases (Visual Studio):} Visual Studio se utilizó específicamente para la creación de diagramas de clases, lo que permitió una representación visual clara de la estructura del proyecto y las relaciones entre las clases.
    
    \item \textbf{Diagramas (VS Code con el Plugin draw.io):} Visual Studio Code, junto con el plugin draw.io, se utilizó para crear varios tipos de diagramas, incluidos los diagramas de casos de uso y diagramas de actividad. Estas representaciones gráficas ayudaron a comprender y comunicar el flujo de trabajo del sistema.
    
    \item \textbf{Memoria en LaTeX:} La documentación y memoria del proyecto se crearon utilizando LaTeX, con el editor VS Code.
    
    \item \textbf{Control de Versiones (Git y GitHub):} Git se utilizó para el control de versiones del código fuente del proyecto. GitHub se empleó como plataforma de alojamiento para el repositorio de Git, lo que facilitó la colaboración en equipo y el seguimiento de cambios.
    
    \item \textbf{Gestión de Tareas (Trello):} Trello se utilizó para la gestión de tareas y la planificación del proyecto. La herramienta permitió organizar y priorizar tareas, así como realizar un seguimiento del progreso de cada elemento del proyecto.
    
    \item \textbf{Burst Compiler:} El Burst Compiler se utilizó para compilar el código C\# en código nativo altamente optimizado, lo que contribuyó significativamente a mejorar el rendimiento en la generación de terrenos.
\end{itemize}

Estas herramientas desempeñaron un papel esencial en la realización exitosa del proyecto, proporcionando las capacidades necesarias para el desarrollo, la documentación, la colaboración en equipo y la optimización de rendimiento.

\subsection{Desafíos y Decisiones de Diseño}

\subsubsection{Desafíos Técnicos}
Durante el diseño y desarrollo del proyecto, se enfrentaron varios desafíos técnicos significativos. Algunos de los desafíos más destacados incluyeron:

\begin{itemize}
    \item \textbf{Optimización de Rendimiento:} Lograr un rendimiento óptimo en la generación procedural de terrenos en tiempo real fue uno de los principales desafíos técnicos. Se implementó el Unity Job System y el Burst Compiler para abordar este desafío.
    
    \item \textbf{Generación Realista:} La generación de terrenos realistas y variados implicó la implementación de algoritmos de ruido Perlin, Simplex y Voronoi, así como la configuración adecuada de parámetros como escalas y octavas.
    
    \item \textbf{Erosión y Características Naturales:} Incorporar algoritmos de erosión para simular características naturales como ríos y cañones fue un desafío adicional.
    
    \item \textbf{Continuidad del terreno:} La continuidad del terreno nuevo que se genera, sin notar la transición entre partes de terreno generadas e integradas al terreno que ya había supuso otro tema técnico que hubo que resolver.
    
    \item \textbf{Corrección de borde en erosión:} Dado que a eorisón se reliza teneindo en cuenta las alturas de cada fragmetnosde terreno y equilibrándolas, porduce incosistencias con las partes de terreno. Resolver esto fue otro desafío.
\end{itemize}

\subsubsection{Decisiones de Diseño}
Las decisiones de diseño desempeñaron un papel fundamental en la arquitectura y funcionalidad del proyecto. Algunas de las decisiones clave incluyeron:

\begin{itemize}
    \item \textbf{Uso del Unity Job System:} Se decidió utilizar el Unity Job System para la paralelización de tareas y optimizar la generación de terrenos, lo que resultó en un rendimiento mejorado.
    
    \item \textbf{Selección de Algoritmos de Ruido:} La elección de implementar algoritmos de ruido Perlin y Simplex permitió generar terrenos realistas y variados con una apariencia natural.
    
    \item \textbf{Erosión para Características Naturales:} La incorporación de algoritmos de erosión en el diseño permitió crear características naturales como ríos y cañones, mejorando la apariencia general del terreno.
    
    \item \textbf{Elección de nave como explorador} La elección de elección de una nave como explorador de terreno se debió a que las irregularidades del terreno para terrenos escarpados podrían complicar la exploración, además de que con un sobrevuelo se pdoría ver mejor el rendimiento de la generació.
\end{itemize}

\subsubsection{Alternativas Consideradas}
Antes de tomar las decisiones de diseño finales, se consideraron varias alternativas, incluyendo:

\begin{itemize}
    \item \textbf{Otras Tecnologías de Generación de Terrenos:} Se evaluaron diferentes tecnologías y enfoques para la generación de terrenos, como el uso de mapas de altura pregenerados versus generación procedural en tiempo real.
    
    \item \textbf{Métodos de Optimización:} Se exploraron diversas técnicas de optimización además del Unity Job System, como el uso de GPU para cálculos intensivos o la paralelización mediante threads manual.
    
    \item \textbf{Otros Algoritmos de Ruido:} Se investigaron algoritmos de ruido alternativos además de Perlin y Simplex para determinar cuáles producirían los resultados deseados y técnicas de generación, como el algoritmo diamante-cuadrado. Pero se optó por el ruido debido a que facilitaba la consistencia entre lso bordes de partes de terreno nuevas generadas
\end{itemize}

\subsection{Planificación del Desarrollo}

\subsubsection{Metodología de Desarrollo}
El proyecto siguió una metodología de desarrollo ágil, lo que permitió una adaptación flexible a medida que se abordaban desafíos técnicos y se tomaban decisiones de diseño. Se realizaron reuniones periódicas de revisión y planificación para ajustar el enfoque según fuera necesario. Cada día se fueron subiendo incrementos de desarrollo al repositorio remoto, gestioando un control de cuáles habían sido las mejroas subidas, cuál era el estado del proyecto y cuáles debían ser los siguientes objetivos.

\subsubsection{Gestión de Tiempo}
La gestión del tiempo se realizó mediante una planificación detallada en Trello. Creando pilas de tareas "por hacer", "en desarrollo", "termiandas" y "mejorables".

\subsubsection{Recursos Necesarios}
Los recursos necesarios para llevar a cabo el desarrollo incluyeron:

\begin{itemize}
    \item Personal de Desarrollo: Para este proyecto el personal fue una única persona que ocupó todos los roles del desarrollo y un product owner que especificaba los requisitos que debía cumplir el proyecto.
    
    \item Hardware: Se utilizó un equipo portátil con procesador i5-11400H con 2.7GHz, 16 GB de RAM, 1TB de memoria en disco y tarjeta gráfica NVidia 3060 con 6GB de RAM. El equipo contaba con windows 10 Home como sistema operativo. 
    
    \item Software: Se requirieron herramientas como Unity3D, JetBrains Rider, Visual Studio, VS Code con el plugin draw.io, LaTeX y el Burst Compiler para el desarrollo y la documentación del proyecto.
\end{itemize}

La gestión eficaz de estos recursos fue fundamental para el éxito del proyecto.


% \chapter{Diseño}
\section{Diseño}

\subsection{Diseño de la Arquitectura}

\subsubsection{Diseño de la Arquitectura del Software}
Describe la arquitectura de software que has diseñado, incluyendo patrones arquitectónicos si los has aplicado.

\subsubsection{Diseño de la Interfaz de Usuario (UI)}
Explica cómo has diseñado la interfaz de usuario, incluyendo esquemas de diseño y experiencia de usuario (UX).

\subsubsection{Diagramas de Clases}
Incluye diagramas de clases que representen la estructura de clases de tu proyecto y sus relaciones.

\subsection{Diseño Detallado}

\subsubsection{Diagramas de Secuencia}
Muestra los diagramas de secuencia que ilustran la interacción entre los componentes clave de tu sistema.

\subsubsection{Diagramas de Actividad}
Incluye diagramas de actividad que representen los flujos de trabajo y procesos en tu aplicación.

\subsubsection{Diagramas de Estado}
Explica los estados y transiciones de los elementos que tienen comportamientos específicos.

\subsubsection{Diseño de Base de Datos}
Si tu proyecto incluye una base de datos, describe su diseño, esquema y relaciones.

\subsection{Decisiones de Diseño}

\subsubsection{Decisiones de Diseño Clave}
Destaca las decisiones de diseño más importantes que afectaron la arquitectura y funcionalidad de tu proyecto.

\subsubsection{Consideraciones de Rendimiento}
Explica las consideraciones de rendimiento que tuviste en cuenta en el diseño.

\subsubsection{Seguridad y Privacidad}
Describe cómo abordaste la seguridad y la privacidad en el diseño de tu aplicación.

\subsection{Prototipado}

\subsubsection{Prototipos de Interfaz}
Muestra prototipos de la interfaz de usuario para visualizar el diseño antes de la implementación.

\subsubsection{Pruebas de Concepto}
Si realizaste pruebas de concepto, explica los resultados y cómo influyeron en el diseño final.

\subsection{Planificación de Desarrollo}

\subsubsection{Cronograma de Desarrollo}
Incluye el cronograma detallado de desarrollo, indicando hitos y fechas clave.

\subsubsection{Asignación de Recursos}
Describe cómo asignaste recursos, incluyendo personal y hardware, para llevar a cabo el diseño y desarrollo.

\subsubsection{Plan de Pruebas}
Explica tu plan de pruebas de diseño y cómo verificarás que el diseño cumple con los requisitos.

\subsubsection{Control de Cambios}
Detalla cómo gestionarás los cambios en el diseño a medida que evolucione el proyecto.

\subsection{Consideraciones Éticas y Legales}

\subsubsection{Ética en el Diseño}
Aborda cualquier consideración ética relacionada con el diseño de tu proyecto.

\subsubsection{Consideraciones Legales}
Describe cualquier consideración legal relacionada con el diseño, como licencias de software y derechos de autor.



% \chapter{Implementación y pruebas}
% \section{Implementación}

\subsection{Entorno de Desarrollo}

\subsubsection{Herramientas de Desarrollo}
Enumera las herramientas de desarrollo que has utilizado, incluyendo el entorno de desarrollo integrado (IDE), editores de código, y otros programas relevantes.

\subsubsection{Lenguajes de Programación}
Describe los lenguajes de programación que has empleado en tu proyecto y su papel en la implementación.

\subsubsection{Frameworks y Bibliotecas}
Menciona cualquier framework o biblioteca de terceros que hayas utilizado y cómo contribuyeron a la implementación.

\subsection{Estructura del Código Fuente}

\subsubsection{Organización de Directorios}
Explica la estructura de directorios de tu proyecto y cómo organizaste los archivos de código fuente.

\subsubsection{Módulos y Componentes Principales}
Identifica los módulos y componentes principales de tu aplicación y cómo se relacionan entre sí.

\subsubsection{Flujo de Datos}
Describe cómo fluyen los datos a través de tu aplicación, desde la entrada hasta la salida.

\subsection{Algoritmos y Técnicas}

\subsubsection{Descripción de Algoritmos Clave}
Explica los algoritmos clave que implementaste en tu proyecto y cómo contribuyen a su funcionamiento.

\subsubsection{Técnicas de Optimización}
Si aplicaste técnicas de optimización, como paralelización o caching, descríbelas y su impacto en el rendimiento.

\subsection{Desarrollo de Características}

\subsubsection{Implementación de Funcionalidades}
Detalla la implementación de las principales funcionalidades de tu proyecto, destacando aspectos relevantes.

\subsubsection{Integración de Componentes}
Explica cómo integras los diferentes componentes y módulos en tu aplicación.

\subsection{Pruebas y Depuración}

\subsubsection{Pruebas Unitarias}
Describe las pruebas unitarias que realizaste y cómo ayudaron a identificar y corregir errores.

\subsubsection{Pruebas de Integración}
Explica cómo llevaste a cabo las pruebas de integración y qué desafíos enfrentaste.

\subsubsection{Depuración y Solución de Problemas}
Detalla el proceso de depuración que seguiste para identificar y resolver problemas en el código.




\chapter{Conclusiones}
% Contenidos del capítulo.
% Las secciones presentadas son orientativas y no representan
% necesariamente la organización que debe tener este capítulo.

\section{Conclusiones}

Este trabajo proporciona a cualquier interesado la capacidad de generar un terreno infinito y dinámico, de aspecto realista, mientras se desplaza por él. Gracias a los diversos tipos de ruido y configuraciones ajustables, esta herramienta ofrece una amplia variedad de combinaciones que permiten crear entornos personalizados con un buen rendimiento.

Esta herramienta es un recurso valioso y versátil, capaz de optimizar tanto el tiempo como los recursos necesarios para generar espacios de juego explorables. Al ser versatil y adaptable, esta herramienta se convierte en una sólida base para proyectos que requieran un terreno de juego personalizado e ilimitado.

\section{Trabajo futuro}
Como futuras líneas de trabajo se propone:

\begin{enumerate}
    \item \textbf{Mejora de la Interacción del Usuario:} Desarrollar una interfaz de usuario más amigable que permita a los creadores de contenido ajustar los parámetros y ver los cambios en tiempo real, lo que facilitaría la creación de terrenos a medida.
    
    \item \textbf{Optimización del Rendimiento:} Continuar trabajando en la optimización del rendimiento para garantizar que la generación de terrenos sea lo más eficiente posible, especialmente en entornos de tiempo real.
    
    \item \textbf{Más Algoritmos de Generación:} Investigar y agregar otros algoritmos de generación de terrenos, lo que brindaría a los usuarios aún más opciones y flexibilidad en la creación de entornos.
    
    \item \textbf{Integración de Simulación Climática:} Incorporar sistemas de simulación climática y de agua que permitan la creación de terrenos que respondan de manera realista a cambios climáticos y eventos naturales.
    
    \item \textbf{Generación Procedural de Biomas:} Expandir la generación procedural para incluir la creación automática de vegetación, animales y criaturas para poblaciones de entornos.
    
    \item \textbf{Herramientas de Escultura 3D:} Agregar herramientas de escultura 3D que permitan a los usuarios modelar y personalizar terrenos con mayor detalle.
    
    \item \textbf{Compatibilidad con Plataformas Externas:} Adaptable a otras plataformas o motores de juegos populares además de Unity para aumentar la accesibilidad de la herramienta.
    
    \item \textbf{Almacenamiento en Disco:} Guardar chunks ya generados a disco, para no tener que generarlos de nuevo sería un añadido que podría ser my útil, en especial para la parte de edición de niveles.
    
    \item \textbf{Generación de Ciudades y Entornos Urbanos:} Ampliar la capacidad de la herramienta para generar ciudades y entornos urbanos completos, lo que sería útil para juegos de mundo abierto y simuladores urbanos.
    
    \item \textbf{Integración de Inteligencia Artificial:} Implementar algoritmos de IA que permitan que los terrenos se adapten y evolucionen de manera dinámica en función de las acciones de los jugadores o las condiciones del juego.
    
    \item \textbf{Compatibilidad con Realidad Virtual (RV) y Realidad Aumentada (RA):} Adaptable para su uso en entornos de RV y RA, lo que brindaría experiencias de juego más inmersivas y realistas.
    
    \item \textbf{Documentación y Comunidad:} Continuar mejorando la documentación y fomentar una comunidad activa de usuarios que compartan sus experiencias y desarrollos utilizando la herramienta.
\end{enumerate}





\pagestyle{appendix}

\appendix
\chapter{Apéndice}
\section*{Cronograma de fases del desarrollo}

En el siguiente diagrama de Gantt se detalla el diseño de la duración de las tareas.

\begin{figure}[H]
    \centering
    \includegraphics[width=1\textwidth]{img/planificación gantt.png}
    \caption{Diagrama de Secuencia de Generación de Terreno.}
\end{figure}


\addcontentsline{toc}{chapter}{Bibliografía}
\bibliographystyle{unsrt}
\bibliography{bib/bibliografia}

\end{document}

%%% Local Variables:
%%% mode: latex
%%% TeX-master: t
%%% End:
