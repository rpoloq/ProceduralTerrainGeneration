\section{Problemática en la Generación Procedural de Terreno}

En este trabajo se pretende mejorar la eficiencia de los videojuegos y la experiencia de los jugadores a través de soluciones destinadas a solucionar problemas relacionados con la generación procedural de terreno en Unity.

De este modo, la gestión de chunks de terreno en el apartado de memoria es fundamental para mejorar la eficiencia, consiguiendo una transición suave según el movimiento de la cámara del jugador y su respectiva proximidad con respecto al entorno. Para conseguir una transición visual adecuada se ha realizado un sistema de "streaming" de terrenos, el cuál carga y descarga los chunks de una manera dinámica mediante técnicas de interpolación y suavizado en las fronteras de los chunks.

Otro objetivo pasa por integrar de la mejor manera el "Job System" de Unity. Aunque no resulta sencillo compaginarlo con otros sistemas del juego y surge la problemática de gestionar distintos trabajos en paralelo, este sistema resulta clave a la hora de mejorar el rendimiento general.

Por otro lado, si bien es cierto que la generación de terreno puede llevarse a cabo correctamente, en ningún momento debe descuidarse el uso de hardware y el tiempo de procesamiento. Para minimizar ambas variables es necesario seleccionar los algoritmos adecuados y optimizarlos correctamente.

Otro desafío pasa por la formación de desarrolladores en el ámbito de la generación de terreno. Para facilitar un aprendizaje adecuado resulta óptimo suministrar herramientas de aprendizaje, que ayuden a los programadores a implementar en sus proyectos técnicas avanzadas.

Aplicar los objetivos mencionados resulta esencial en la mayoría de proyectos, sobre todo en aquellos que utilicen muchos recursos. De este modo, la mejora fundamental pasa por la eficiencia de los algoritmos en función de memoria y rendimiento, la correcta integración de efectos en las escenas y su facilidad y versatilidad de uso.

% \newpage