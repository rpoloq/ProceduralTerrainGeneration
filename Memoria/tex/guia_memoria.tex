Este es el prefacio de la memoria. A continuación, se presenta una breve descripción de los capítulos que componen esta memoria:

\begin{description}
  \item[Capítulo 1. MARCO TEÓRICO. INTRODUCCIÓN.] En este capítulo se realiza una introducción a la motivación por la cual se ha llevado a cabo este proyecto, las técnicas más comunes que se utilizan para la generación procedural, el contexto en el que se desarrolla e importancia de este.

  \item[Capítulo 2. PLANTEAMIENTO DEL PROBLEMA.] En este capítulo se indican los obstáculos que plantea la generación procedural, el manejo de la memoria, de la optimización temporal, la continuidad del terreno, así como el realismo de los resultados.

  \item[Capítulo 3. ESTADO DEL ARTE.] En este capítulo se analiza el estado del arte de soluciones de generación procedural con objetivos similares a la propuesta de este proyecto tras realizar una revisión bibliográfica.

  \item[Capítulo 4. MARCO METODOLÓGICO. MATERIAL Y MÉTODOS.] En este capítulo se detalla el diseño y el procedimiento de la herramienta que se desarrolla en este proyecto que permite compararla herramienta frente a otras ya existentes.

  \item[Capítulo 5. DESARROLLO.] En este capítulo se detalla el desarrollo software de la herramienta, el diseño y la manera en la que se implementa en el motor Unity para ser usado por desarrolladores.

  \item[Capítulo 6. RESULTADOS Y DISCUSIÓN.] En este capítulo se recogen todos los resultados de las pruebas establecidas para la evaluación de los algoritmos que crean los terrenos de la herramienta. Se describe el conjunto de parámetros empleados y las métricas establecidas para la evaluación de la variable de interés y para la validación de los resultados, tanto visuales como de rendimiento. Se analizan los resultados obtenidos y se identifican los factores y limitaciones que han podido influir en ellos.

  \item[Capítulo 7. CONCLUSIONES Y LÍNEAS FUTURAS.] Se recogen las conclusiones extraídas a partir del diseño, desarrollo y experimentación del método propuesto para la elaboración de la herramienta. Se describen futuras líneas de trabajo que complementen el desarrollo alcanzado y se indica la aplicación de este proyecto en el ámbito del desarrollo de contenido multimedia.
\end{description}