Este es el prefacio de la memoria. A continuación, se presenta una breve descripción de los capítulos que componen esta memoria:

\begin{description}
  \item[Capítulo 1. INTRODUCCIÓN.] En este capítulo se realiza una introducción a la motivación por la cual se ha llevado a cabo este proyecto, las técnicas más comunes que se utilizan para la generación procedural, el contexto en el que se desarrolla e importancia de este.

  \item[Capítulo 2. PLANTEAMIENTO DEL PROBLEMA.] En este capítulo se indican los obstáculos que plantea la generación procedural, el manejo de la memoria, de la optimización temporal, la continuidad del terreno, así como el realismo de los resultados.
  
  \item[Capítulo 3. OBJETIVOS.] En este capítulo se indican los objetivos que se pretenden alcnazar en el proyecto y cómo se pretende conseguirlos.

  \item[Capítulo 4. ESTADO DEL ARTE.] En este capítulo se analiza el estado del arte de soluciones de generación procedural con objetivos similares a la propuesta de este proyecto tras realizar una revisión bibliográfica.

  \item[Capítulo 5. DESARROLLO DE LA SOLUCIÓN.] En este capítulo se detalla el análisis, diseño e implementación de la todos los componentes, sistemas y algoritmos empleados para la generación de terreno procedimental en Unity.

  \item[Capítulo 6. ANÁLISIS DE RESULTADOS.] En este capítulo se recogen todos los resultados de las pruebas establecidas para la evaluación de los algoritmos que crean los terrenos de la herramienta. Se describe el conjunto de parámetros empleados y las métricas establecidas para la evaluación de la variable de interés y para la validación de los resultados, tanto visuales como de rendimiento. Se analizan los resultados obtenidos y se identifican los factores y limitaciones que han podido influir en ellos.

  \item[Capítulo 7. CONCLUSIONES Y LÍNEAS FUTURAS.] Se recogen las conclusiones extraídas a partir del diseño, desarrollo y experimentación del método propuesto para la elaboración de la herramienta. Se describen futuras líneas de trabajo que complementen el desarrollo alcanzado y se indica la aplicación de este proyecto en el ámbito del desarrollo de contenido multimedia.
\end{description}