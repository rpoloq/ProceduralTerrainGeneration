% Contenidos del capítulo.
% Las secciones presentadas son orientativas y no representan
% necesariamente la organización que debe tener este capítulo.

\section{Introducción}

Con el paso de los años, los videojuegos han ido evolucionando al igual que toda su industria. Esta evolución se debe a la búsqueda de encontrar nuevos mundos virtuales que impresionen y puedan producir efectos muy enriquecedores a nivel visual. Por ello, los desarrolladores se han visto en la necesidad de indagar en esta materia, lo que nos lleva a hablar sobre la generación procedural de terrenos, por ser la herramienta más efectiva para poder cubrir la demanda de los jugadores.

Este trabajo de fin de máster (TFM) cubre el apasionante ámbito de la generación de terreno procedimental en Unity, uno de los motores de desarrollo de videojuegos comúnmente utilizado por la industria y también por desarrolladores independientes. La meta de este proyecto es diseñar y desarrollar herramientas avanzadas para que los desarrolladores de juegos puedan generar de manera convincente y eficiente terrenos de juego.

No quisera seguir con la Introducción de este TFM sin mencionar a SebastianLague, creador de contenido de gráficos por computador que me ha brindado ayuda con sus tutoriales y proyectos que e han permitdo crear la base para este proyecto. También se ha comprobado que los jugadores buscan experiencias únicas pero auténticas, y la generación procedural de terrenos tiene la solución.

Los algoritmos basados en ruido son la historia de la generación procedural; entre todos ellos destacó el Perlin Noise, que a pesar de dar resultados muy exitosos, no era suficiente para crear esa autenticidad que los jugadores demandaban. El Simplex Noise supuso un avance importante, pero lo realmente relevante en este ámbito fue la aparición de algoritmos de erosión. Estos pueden simular características naturales muy convincentes al tratarse de procesos climáticos o geológicos, como montañas o ríos. Esto motivó a los jugadores a tener experiencias más fluidas y expansivas, mejorando de manera muy enriquecedora las sensaciones del jugador mientras estaba inmerso en el videojuego.

La generación procedural de terrenos abarca diversos sectores; uno de los más utilizados, como ya hemos dicho, es la creación de paisajes naturales. Sin embargo, su producción no acaba aquí. También se utiliza para añadir complejidad a la creación de algunas ciudades o para establecer misiones con contenido diferente, así como para la reproducción de los personajes que complementan el mundo virtual. Cabe destacar que otros aspectos, como la música, los efectos, el tiempo y los recursos, también se ven beneficiados por la generación procedural.

En los últimos años, se ha creado una tendencia a utilizar de manera mucho más frecuente la generación procedural de terrenos en la elaboración de la mayoría de los videojuegos, lo que ha supuesto un notable cambio, ya que se han alcanzado a crear mundos infinitos.

Para la generación procedural, uno de los retos más difíciles, particularmente en entornos de mundo abierto, es la gestión de trozos de terreno. Estos son pequeños pedazos de terreno que se generan y almacenan en la memoria para descargarse y cargarse dinámicamente mientras el jugador está a mitad de la partida y se va moviendo por el mundo virtual.

Este enfoque depende de diversos factores: la proximidad del jugador, la dirección de su movimiento y su campo de visión.

Una de las cosas a tener en cuenta es la posición actual de la cámara del jugador, ya que esta servirá para establecer qué "chunks" deben ser cargados y renderizados, a través de unos algoritmos de detección de colisiones y de visibilidad. Estos algoritmos tienen una gran trascendencia, ya que permiten disminuir la carga en la unidad central del procesamiento (CPU) y la unidad de procesamiento gráfico (GPU). Cabe destacar también las técnicas de nivel de detalle "LOD" que se encargan de ajustar la resolución y el nivel de detalle del terreno en tiempo real en función de la distancia entre la cámara y el terreno circundante; es decir, cuando la cámara está alejada, se incrementa el nivel de detalle y viceversa. Esto se hace para mostrar texturas de mayor calidad y para mantener los recursos de hardware.

El Job System de Unity ha tenido una gran relevancia en este campo, ya que permite la optimización de algunos procesos en la CPU, como por ejemplo, la modificación de mallas de terreno.

\section{Motivación}

El desarrollo de la herramienta de generación procedural de terrenos en Unity se enmarca en un contexto dinámico y desafiante, impulsado por diversas motivaciones que abordan necesidades y aspiraciones cruciales.

La industria de los videojuegos se encuentra en constante evolución, y la búsqueda incesante de experiencias más inmersivas y visualmente impactantes impulsa a los desarrolladores a explorar nuevas formas de crear mundos virtuales. La generación procedural de terrenos se erige como una respuesta a esta demanda creciente. Esta técnica no solo permite satisfacer las expectativas de los jugadores modernos en términos de autenticidad y sorpresa, sino que también agiliza el proceso de desarrollo de videojuegos al proporcionar herramientas eficientes para crear entornos expansivos y convincentes.

La gestión eficiente de chunks de terreno, la optimización del "Job System" de Unity y la elección de algoritmos adecuados son áreas que requieren innovación y soluciones creativas. Esta necesidad de abordar y superar desafíos técnicos en el campo de la generación procedural de terrenos motiva el desarrollo de esta herramienta. La oportunidad de enfrentar estos desafíos y crear una herramienta que marque la diferencia en la industria es una fuerza impulsora clave.

Este proyecto se enmarca en el contexto de una industria de videojuegos en constante cambio y una demanda creciente de experiencias inmersivas. La búsqueda de soluciones técnicas innovadoras y la aspiración de contribuir al crecimiento de la comunidad de desarrolladores son las motivaciones que guían este esfuerzo. La generación procedural de terrenos en Unity tiene el potencial de transformar la forma en que se crean mundos virtuales, y este proyecto se esfuerza por lograr precisamente eso.
\newpage
