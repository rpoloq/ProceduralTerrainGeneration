% Contenidos del capítulo.
% Las secciones presentadas son orientativas y no representan
% necesariamente la organización que debe tener este capítulo.

\section{Introducción}

La industria de los videojuegos ha sido testigo de una transformación excepcional en los últimos años. La búsqueda constante de experiencias más inmersivas y visualmente impresionantes ha llevado a los desarrolladores a explorar nuevas formas de crear mundos virtuales. En este contexto, la generación procedural de terrenos se ha convertido en una herramienta esencial para satisfacer las demandas de los jugadores cada vez más exigentes y hambrientos de autenticidad.

Este Trabajo de Fin de Máster (TFM) se adentra en el emocionante campo de la generación procedural de terrenos en Unity, uno de los motores de desarrollo de videojuegos más utilizados tanto por la industria como por desarrolladores independientes. El objetivo principal de este proyecto es diseñar y desarrollar una herramienta avanzada que permita a los creadores de juegos generar terrenos de manera eficiente y convincente.

Los datos disponibles, incluidos los proporcionados por fuentes como Eurostat y otros informes, indican un crecimiento constante en la industria de los videojuegos. A medida que la audiencia se expande, también lo hacen sus expectativas. Los jugadores contemporáneos buscan experiencias de juego que sean únicas y estimulantes. En este contexto, la generación procedural de terrenos se plantea como una solución que puede hacer posible la creación de mundos tanto auténticos como sorprendentes.

La historia de la generación procedural de terrenos comenzó con la implementación de algoritmos basados en ruido, entre los que destaca el renombrado Perlin Noise, desarrollado por Ken Perlin en la década de 1980. A pesar de que estos enfoques ofrecieron resultados impresionantes para su época, a menudo carecían de la autenticidad y el realismo que los jugadores modernos esperan en la actualidad. Con el aumento de la demanda de experiencias de juego más inmersivas y visualmente impactantes, se volvió crucial mejorar las técnicas de generación de terrenos.

Un avance destacado en este ámbito fue la introducción del Simplex Noise, una mejora significativa respecto al Perlin Noise, que mantuvo la capacidad de generar terrenos naturales, pero con un rendimiento más eficiente. No obstante, lo que realmente marcó un hito en la generación procedural de terrenos fue la incorporación de algoritmos de erosión. Estos algoritmos simulan procesos geológicos y climáticos, lo que resulta en terrenos con características naturales más convincentes, como montañas, cañones y ríos. Esta adición permitió crear mundos virtuales más realistas y creíbles, lo que contribuyó a elevar la calidad general de los juegos.

La creciente popularidad de los mundos abiertos en los videojuegos presentó un desafío significativo: generar terrenos continuos y sin interrupciones notables cuando los jugadores exploran los límites del mundo virtual. A medida que los algoritmos mejoraron, los juegos pudieron ofrecer experiencias de juego más fluidas y expansivas, lo que aumentó la inmersión del jugador y su sensación de exploración sin restricciones.

La influencia de la generación procedural de terrenos no se limita únicamente a la creación de paisajes virtuales. Ha dejado una huella indeleble en otros aspectos de la generación procedural en la industria de los videojuegos. Esto incluye la generación de ciudades completas en juegos de mundo abierto, la creación de misiones y contenido diverso que aumenta la rejugabilidad, así como la generación de personajes y criaturas que dan vida a los mundos virtuales. Además, ha influido en la generación de texturas y elementos artísticos, permitiendo un ahorro significativo de tiempo y recursos en el desarrollo de juegos y animaciones. La música y los efectos de sonido también se benefician de la generación procedural, adaptando la banda sonora y la atmósfera del juego en tiempo real para acompañar la acción.

En la última década, hemos sido testigos de un aumento significativo en la adopción de la generación procedural de terrenos en la industria de los videojuegos. Esto marca un cambio notable en la forma en que los desarrolladores abordan la creación de entornos virtuales. Esta adopción creciente se ha visto impulsada por varios factores clave, como la capacidad de generar mundos virtualmente infinitos sin incurrir en costos prohibitivos de almacenamiento. Además, proporciona una experiencia de juego única en cada partida, lo que aumenta la rejugabilidad y la longevidad de un título. Esta estrategia también es esencial en la creación de mundos abiertos sin pantallas de carga notables entre escenarios o biomas, lo que mejora la inmersión del jugador.

La evolución de la generación procedural de terrenos ha estado intrínsecamente ligada a la constante innovación en algoritmos y técnicas. En los últimos años, se han desarrollado y refinado una variedad de enfoques que van desde la generación basada en ruido, como el Perlin Noise y el Simplex Noise, hasta técnicas más avanzadas que incorporan algoritmos de erosión y simulaciones físicas para lograr terrenos aún más realistas.

Uno de los desafíos cruciales en la generación procedural de terrenos, especialmente en entornos de mundo abierto, es la gestión eficiente de los "chunks" de terreno, pequeñas porciones de terreno que se generan y almacenan en memoria para luego ser cargadas y descargadas dinámicamente mientras el jugador se mueve a través del mundo virtual. Este proceso es esencial para mantener un rendimiento óptimo y evitar la carga excesiva de recursos en la memoria del sistema. Además, es fundamental asegurarse de que la transición entre diferentes chunks sea fluida y sin discontinuidades notorias, lo que garantiza una experiencia de juego inmersiva. En el contexto de la generación procedural de terrenos, se ha consolidado una técnica comúnmente empleada denominada "sistema de streaming" de terrenos. Este enfoque se encarga de gestionar la carga y descarga de fragmentos o "chunks" de terreno de manera dinámica en función de diversos factores, como la proximidad del jugador, la dirección de su movimiento y su campo de visión. Uno de los principales objetivos de este sistema es garantizar una transición fluida entre chunks adyacentes, evitando discontinuidades notorias que puedan afectar negativamente a la cohesión y la inmersión en el mundo virtual.

Además, se emplean algoritmos de detección de colisiones y de visibilidad para determinar qué chunks deben ser cargados y renderizados según la posición actual de la cámara del jugador. Estos algoritmos son cruciales para optimizar el rendimiento del juego, ya que permiten reducir la carga en la unidad central de procesamiento (CPU) y la unidad de procesamiento gráfico (GPU). En consecuencia, solo se procesan y muestran los chunks que son relevantes y visibles desde la perspectiva del jugador en ese momento particular. Esta estrategia de optimización contribuye significativamente a lograr una experiencia de juego fluida y eficiente en términos de recursos.

En lo que respecta al movimiento de la cámara del jugador y su influencia en la generación del terreno, se implementan técnicas de "nivel de detalle" (LOD, por sus siglas en inglés) dinámico. Esta técnica se encarga de ajustar la resolución y el nivel de detalle del terreno en tiempo real en función de la distancia entre la cámara y el terreno circundante. Cuando la cámara se acerca a un chunk de terreno en particular, se aumenta el nivel de detalle, lo que implica mostrar texturas de mayor calidad y una geometría más detallada. Por otro lado, cuando la cámara se aleja, se reduce el nivel de detalle para conservar recursos de hardware y mantener un rendimiento óptimo.

En este contexto, el uso del Job System de Unity ha ganado relevancia. Permite la optimización de procesos intensivos en CPU, como la modificación de mallas de terreno, lo que resulta en una mejora significativa del rendimiento en juegos que implementan generación procedural de terrenos.

\section{Motivación}

El desarrollo de la herramienta de generación procedural de terrenos en Unity se enmarca en un contexto dinámico y desafiante, impulsado por diversas motivaciones que abordan necesidades y aspiraciones cruciales.

La industria de los videojuegos se encuentra en constante evolución, y la búsqueda incesante de experiencias más inmersivas y visualmente impactantes impulsa a los desarrolladores a explorar nuevas formas de crear mundos virtuales. La generación procedural de terrenos se erige como una respuesta a esta demanda creciente. Esta técnica no solo permite satisfacer las expectativas de los jugadores modernos en términos de autenticidad y sorpresa, sino que también agiliza el proceso de desarrollo de videojuegos al proporcionar herramientas eficientes para crear entornos expansivos y convincentes.

La gestión eficiente de chunks de terreno, la optimización del "Job System" de Unity y la elección de algoritmos adecuados son áreas que requieren innovación y soluciones creativas. Esta necesidad de abordar y superar desafíos técnicos en el campo de la generación procedural de terrenos motiva el desarrollo de esta herramienta. La oportunidad de enfrentar estos desafíos y crear una herramienta que marque la diferencia en la industria es una fuerza impulsora clave.

Este proyecto se enmarca en el contexto de una industria de videojuegos en constante cambio y una demanda creciente de experiencias inmersivas. La búsqueda de soluciones técnicas innovadoras y la aspiración de contribuir al crecimiento de la comunidad de desarrolladores son las motivaciones que guían este esfuerzo. La generación procedural de terrenos en Unity tiene el potencial de transformar la forma en que se crean mundos virtuales, y este proyecto se esfuerza por lograr precisamente eso.
\newpage
