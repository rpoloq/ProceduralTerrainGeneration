% Contenidos del capítulo.
% Las secciones presentadas son orientativas y no representan
% necesariamente la organización que debe tener este capítulo.

\section{Conclusiones}

Este trabajo proporciona a cualquier interesado la capacidad de generar un terreno infinito y dinámico, de aspecto realista, mientras se desplaza por él. Gracias a los diversos tipos de ruido y configuraciones ajustables, esta herramienta ofrece una amplia variedad de combinaciones que permiten crear entornos personalizados con un buen rendimiento.

Esta herramienta es un recurso valioso y versátil, capaz de optimizar tanto el tiempo como los recursos necesarios para generar espacios de juego explorables. Al ser versatil y adaptable, esta herramienta se convierte en una sólida base para proyectos que requieran un terreno de juego personalizado e ilimitado.

\section{Trabajo futuro}
Como futuras líneas de trabajo se propone:

\begin{enumerate}
    \item \textbf{Mejora de la Interacción del Usuario:} Desarrollar una interfaz de usuario más amigable que permita a los creadores de contenido ajustar los parámetros y ver los cambios en tiempo real, lo que facilitaría la creación de terrenos a medida.
    
    \item \textbf{Optimización del Rendimiento:} Continuar trabajando en la optimización del rendimiento para garantizar que la generación de terrenos sea lo más eficiente posible, especialmente en entornos de tiempo real.
    
    \item \textbf{Más Algoritmos de Generación:} Investigar y agregar otros algoritmos de generación de terrenos, lo que brindaría a los usuarios aún más opciones y flexibilidad en la creación de entornos.
    
    \item \textbf{Integración de Simulación Climática:} Incorporar sistemas de simulación climática y de agua que permitan la creación de terrenos que respondan de manera realista a cambios climáticos y eventos naturales.
    
    \item \textbf{Generación Procedural de Biomas:} Expandir la generación procedural para incluir la creación automática de vegetación, animales y criaturas para poblaciones de entornos.
    
    \item \textbf{Herramientas de Escultura 3D:} Agregar herramientas de escultura 3D que permitan a los usuarios modelar y personalizar terrenos con mayor detalle.
    
    \item \textbf{Compatibilidad con Plataformas Externas:} Adaptable a otras plataformas o motores de juegos populares además de Unity para aumentar la accesibilidad de la herramienta.
    
    \item \textbf{Almacenamiento en Disco:} Guardar chunks ya generados a disco, para no tener que generarlos de nuevo sería un añadido que podría ser my útil, en especial para la parte de edición de niveles.
    
    \item \textbf{Generación de Ciudades y Entornos Urbanos:} Ampliar la capacidad de la herramienta para generar ciudades y entornos urbanos completos, lo que sería útil para juegos de mundo abierto y simuladores urbanos.
    
    \item \textbf{Integración de Inteligencia Artificial:} Implementar algoritmos de IA que permitan que los terrenos se adapten y evolucionen de manera dinámica en función de las acciones de los jugadores o las condiciones del juego.
    
    \item \textbf{Compatibilidad con Realidad Virtual (RV) y Realidad Aumentada (RA):} Adaptable para su uso en entornos de RV y RA, lo que brindaría experiencias de juego más inmersivas y realistas.
    
    \item \textbf{Documentación y Comunidad:} Continuar mejorando la documentación y fomentar una comunidad activa de usuarios que compartan sus experiencias y desarrollos utilizando la herramienta.
\end{enumerate}


