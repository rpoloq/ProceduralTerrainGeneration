% Contenidos del capítulo.
% Las secciones presentadas son orientativas y no representan
% necesariamente la organización que debe tener este capítulo.


\section{Objetivos Generales y Específicos}

Este proyecto de generación procedural de terrenos en Unity se enfrenta al desafío de diseñar y desarrollar una herramienta innovadora que aborde las necesidades de la industria de los videojuegos. Los objetivos generales y específicos se centran en la creación de una solución tecnológica efectiva y en la evaluación de su impacto en el proceso de desarrollo de videojuegos. Aquí se desarrollan los objetivos que se deben lograr para llevar a cabo la herramienta conforme se desea:

\begin{enumerate}
    \item \textbf{Diseñar y Desarrollar de la Herramienta:} El objetivo fundamental es crear una solución tecnológica efectiva que permita a los desarrolladores de videojuegos generar terrenos de manera eficiente y convincente. Esta herramienta debe ser capaz de abordar los desafíos técnicos de la generación procedural de terrenos y proporcionar una interfaz intuitiva para los usuarios. 

    \item \textbf{Analizar el Impacto y la Eficiencia de la Herramienta:} Más allá de su desarrollo, es esencial evaluar el aporte real de esta herramienta en el proceso de creación de videojuegos. Esto implica medir la capacidad de la herramienta para optimizar la generación de terrenos y su influencia en la calidad de los juegos resultantes.

    \begin{enumerate}[label=\Alph*)]
        \item \textbf{Investigación y Selección de Algoritmos y Técnicas:} Se realizará una investigación exhaustiva para analizar y seleccionar los algoritmos y técnicas más adecuados para la generación procedural de terrenos en Unity.

        \item \textbf{Desarrollo de la Interfaz y Experiencia de Usuario:} El diseño de la interfaz de usuario y la experiencia de usuario son elementos cruciales para la herramienta. Se aplicará una estrategia de diseño centrada en el usuario.

        \item \textbf{Optimización del Rendimiento:} La generación procedural de terrenos puede ser intensiva en términos de rendimiento. Se establecerán estrategias de optimización para garantizar que la herramienta funcione de manera eficiente en diferentes entornos de desarrollo.

        \item \textbf{Evaluación de la Herramienta en un Entorno Real:} Se llevará a cabo un piloto experimental para evaluar la eficacia y eficiencia de la herramienta en un contexto de desarrollo de videojuegos.
    \end{enumerate}
\end{enumerate}

\newpage

