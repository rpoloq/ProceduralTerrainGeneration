% Contenidos del capítulo
% Las secciones presentadas son orientativas y no representan
% necesariamente la organización que debe tener este capítulo.

\section{Requisitos}
% Requisitos del sistema
En cuanto a los requisitos de este proyecto se hará distinción entre requisitos funcionales y requisitos no funcionales. Los requisitos definirán cuales son los objetivos en cuanto al funcionamiento del complemento, así como las características que este deberá tener para su correcto uso por parte de los usuarios.
\subsection{Requisitos funcionales}
Los requisitos funcionales concretan cual debe ser el comportamiento del complemento de forma específica.

\begin{itemize}[label={}]
	
	\item 1. El usuario de poder desplegar y plegar tanto los subpaneles de los efectos como el panel principal.
	\item 2. Cada efecto debe ser configurable a través de parámetros que darán un resultado u otro en función de los valores introducidos en dichos parámetros.
	\item 3. Cada efecto debe ser capaz de ajustarse a la escena en la que es integrado gracias a los parámetros introducidos.
	\item 4. Cada efecto podrá ser reproducido tantas veces como el usuario desee. 
	\item 5. Las modificaciones en los parámetros del efecto deben verse reflejados en este siempre que sea posible en el editor para su comprobación de su resultado.
	\item 6. El complemento debe permitir insertar keyframes y borrarlos en los fotogramas que el usuario tenga seleccionado para el efecto deseado.
	\item 7. Las modificaciones realizadas a una instancia de un efecto deberán verse reflejadas en esa instancia únicamente.
\end{itemize}

\subsection{Requisitos no funcionales}
Los requisitos funcionales concretan cual debe ser el comportamiento del complemento de forma específica.

\begin{itemize}[label={}]
	\item 1. El complemento debe poderse instalar en Blender para versiones de la 2.9 en adelante. 
	\item 2. Los efectos deben poder ejecutarse sin un gran consumo de memoria ni de tiempo de ejecución para que puedan ser ejecutados por el mayor número de equipos posible.
	\item 3. Los efectos deben ser los más adaptables posibles para que se puedan usar para el mayor número de propósitos posible.
	\item 4. Los efectos deben ser versátiles para poder usarlos en todo tipo de escenas.
	\item 5. La implementación del complemento será sobre lenguaje Python.
	\item 6. Los nombres de los parámetros para la implementación del efecto serán claros dando a entender qué es lo que hacen claramente.
	\item 7. El usuario podrá introducir los parámetros de manera sencilla e intuitiva para generar el efecto que desee.
	\item 8. El código será público y podrá ser visto por los usuarios que lo utilicen.
\end{itemize}

\newpage

\section{Especificaciones}
% Especificación del sistema a partir de lo recogido en los requisitos

\section{Costes}
% Costes temporales y económicos

\section{Riesgos}
% Riesgos que pueden incurrir en el desarrollo del sistema

\section{Viabilidad}
% Viabilidad del proyecto presentado
